\documentclass{book}

\usepackage{enumitem}
\usepackage{amsmath}
\usepackage{amsfonts}
\usepackage[backref,pdfpagemode=FullScreen,colorlinks=true]{hyperref}

\numberwithin{equation}{section}

\begin{document}

\title{Solutions to \\
\emph{A Path to Combinatorics for Undergraduates} \\
by Titu Andreescu and Zuming Feng}
\author{\href{https://github.com/cmp5au}{Colin Parker}
and \href{http://profiles.google.com/quantumelixir}{quantumelixir}
(for some of Chapter 1)}
\maketitle

\chapter*{small note}
I am writing this solution manual mostly to motivate myself to finish
all the problems from this book. As a useful side-effect, I also hope
that it would be of use to people who are either stuck on a problem or
want to check their answers against someone else's. However, I am doing
no great good, and my selfish motives take precedence of course. I make
no claims of correctness and shamelessly admit that there might well be
serious mistakes in the answer or in the solution to a problem. I
encourage you to send me corrections by sending pull-requests through
\href{http://github.com/cmp5au/path-to-combinatorics}{github}.
If you are not well-versed with such marvels of the 21st century, you
can still send me an \href{mailto:colin.parker@gmail.com}{email}
informing me of the error.

\vspace*{\fill}
\begin{center}
\href{http://creativecommons.org/licenses/by-nc-sa/3.0/}{by-nc-sa 3.0}
\end{center}

\newpage
\vspace*{\fill}
\begin{flushright}
\emph{If people do not believe that mathematics is simple, it is only
because they do not realize how complicated life is.}\\
--John von Neumann
\end{flushright}

\tableofcontents

\vspace*{\fill}
\emph{Combinatorics is an honest subject. No ad\`eles, no sigma-algebras.
You count balls in a box, and you either have the right number or you
haven't. You get the feeling that the result you have discovered is
forever, because it’s concrete. Other branches of mathematics are not so
clear-cut. Functional analysis of infinite-dimensional spaces is never
fully convincing; you don't get a feeling of having done an honest day's
work. Don't get the wrong idea – combinatorics is not just putting balls
into boxes. Counting finite sets can be a highbrow undertaking, with
sophisticated techniques.}\\
\begin{flushright}
--Gian-Carlo Rota
\end{flushright}
\vspace*{\fill}

\chapter{Addition or Multiplication}

\begin{enumerate}[label={1.\arabic*}]

\item
The total number of functions from $$\{1, 2, \dots, 1998\} \to \{2000, 2001,
2002, 2003\}$$ is $4^{1998}$. For each such function $f$, the sum
$f(1)+f(2)+\dots+f(1998)$ might be either odd or even. If the sum is
even, then $f(1999)$ can be assigned to either of $\{2001, 2003\}$, and if
odd, then $f(1999)$ can be assigned to either of $\{2000, 2002\}$. Hence,
there are two possibilities in each case and required number of
functions is $2\times4^{1998}=2^{3997}$

Colin's Alt: The total number of functions from
$$\{1, 2, \dots, 1999\} \to
\{2000, 2001, 2002, 2003\}$$ is $4^{1999}$. By a similar argument to what is
given above, half of these sum to even and half sum to odd. Therefore the
number of functions is $\frac{4^{1999}}{2} = 2^{3997}$

\item
Let $ab=10a+b$ be the two-digit positive number. For divisibility by
each of it's digits, we have the divisbility conditions: $a | b$, and
$b | 10a$. Taking $b=q_1a$ and $10a=q_2b$, we have $q_1q_2=10$. The only
integer solutions $(q_1,q_2)$ for which are:
$(1,10),(2,5),(5,2),(10,1)$. For each solution, applying the range
constraints $0<a,b<10$, we get the final answer $9 + 4 + 1 + 0 = 14$

\item
We have a bijection between the number of paths from the letter C in
the first row to each R in the last row, and the number of ways to
form the word COMPUTER. With the row index as $n$ and the column index
as $r$, we count the number of paths from the first row to the $(n,
r)$-th element using the recursion ${n \choose r} = {{n-1} \choose {r-1}}
+ {{n-1} \choose r}$. Hence, coefficients in the last row are exactly the binomial
coefficients and we have the answer ${7 \choose 0} + {7 \choose 1} + \dots +
{7 \choose 7} = 2^7 = 128$

Colin's Alt: Any path results in a unique way of spelling COMPUTER, and
there are 7 binary choices to be made. Therefore the number of ways is $2^7$

\item
From the given, we know that $n(A) = n(B) = 2^{100}$. As $n(C) = 2^k$, $n(A
\cup B \cup C) = 2^l$ for some nonnegative integers $k, l$, the only values
that solve the problem equation are $k = 101, l = 102$. Therefore \newline
$|C| = 101,
|A \cup B \cup C| = 102$. We use the equation $$|S_1 \cup S_2| = |S_1| + |S_2| 
- |S_1 \cap S_2|$$ cyclically on $A$, $B$, and $C$. From here, we know $100 
\leq |A \cup B| \leq 102$, and $101 \leq |A \cup C|, |B \cup C| \leq 102$;
using the above equation yields $|A \cap B| \geq 98$, $|A \cap C|, |B \cap
C| \geq 99$
Finally, using the 3-element PIE: $$|A \cap B \cap C| = |A \cup B \cup C| -
(|A| + |B| + |C|) + |A \cap B| + |A \cap C| + |B \cap C|$$ 
$$\geq 102 - 301 + 98 + 99 + 99 = 97$$.

\item
The number of ways to order $n$ distinct objects around a circle is
$(n-1)!$. Within each pair, the twins can be permuted in two ways. Thus,
the total number of ways to arrange the twins is $(12-1)! \times 2^{12}$

\item
Colin's Alt: The 4-digit integer and its consecutive number don't require
carrying when it is of the form $1xyz$, where $x, y \in \{0, 1, 2, 3, 4\}$
and $z \in \{0, 1, 2, 3, 4, 9\}$. In addition, the 4-digit integers of the
form $1w99$ where $w \in \{0, 1, 2, 3, 4, 9\}$ don't require carrying when
added to their subsequent. The total number is $5^2 \cdot 6 + 6 = 156$

\item
Let the number of black and white marbles in the first and second box be
$b_1, w_1$ and, $b_2, w_2$ respectively. We are given that $b_1 + w_1 +
b_2 + w_2 = 25$. $P(\text{both marbles are black}) = \frac{b_1}{b_1+w_1}
\times \frac{b_2}{b_2+w_2} = \frac{27}{50}$. So have the following
condition: $(b_1+w_1)+(b_2+w_2)=25$ and $(b_1+w_1)(b_2+w_2)|50$, to
which the only solutions $(b_1+w_1,b_2+w_2)$ are $(5,20), (10,15)$
(discarding permutations). When $(b_1+w_1,b_2+w_2)=(5,20)$,
$b_1b_2=54=2\cdot3^3$ and the only possible solution to the number of
black balls in the first and second box is $(3,2\cdot3^2)$. This yields
$\frac{w_1w_2}{(b_1+w_1)(b_2+w_2)}=\frac{(5-3)\cdot(20-2\cdot3^2)}{5\cdot20}=\frac{1}{25}$.
When $(b_1+w_1,b_2+w_2)=(l0,15)$, $b_1b_2=81=3^4$ and the only possible
solution to the number of black balls in the first and second box is
$(3^2,3^2)$. This yields $\frac{w_1w_2}{(b_1+w_1)(b_2+w_2)} =
\frac{(10-3^2)\cdot(15-3^2)}{10\cdot15} = \frac{1}{25}$. Thus, in each
case, the answer is $\frac{1}{25}$

N.C. This is a bad problem because it is ambiguous and a lucky chance
resolves the ambiguity, something like how $\frac{49}{98} = \frac{4}{8}$,
making it appear as though the $9$s cancel each other.

\item
Colin's Alt: Placing the girls first, there are $9!$ ways to do so around
a circle. The boys then can be placed in 10 spots, for a total of $10 \cdot
9 \cdot 8 \cdot 7$ ways of placing the boys. The girls and boys are placed
independently, giving $10! \cdot 9 \cdot 8 \cdot 7$ ways in total.

Incorrect solution:
Separating the $10$ girls by $10$ blank placeholders, we can choose $4$
places for the boys in $10 \choose 4$ ways. Permuting the girls and
boys, after choosing where the boys sit, we get a total of
${10\choose4}10!4!$ ways.

\item
Each element of $S=\{1, 2, \dots, 2003\}$ goes to one or two sets from
the ordered triple. There are ${3 \choose 1} + {3 \choose 2} = 6$ ways
of choosing one or two sets.
Thus, the total number of ways of assigning all the
elements from $S$ is $6^{2003}$.

\item
The question is equivalent to finding the number of increasing
arithmetic progressions in the set $S=\{1, 2, \dots, 2000\}$.  Suppose
we choose $1$ as the first term. It is clear that we can only choose the
odd numbers $\{3, 5, \dots, 1999\}$ for the middle term to be an integer
and lie in $S$.  There are exactly $1000$ such arithmetic progressions
beginning with $1$.  We argue similarly when the first term is $2, 3,
\dots, 1998$. Thus, the total number of ways to select a three term
increasing arithmetic progression from $S$ is $\underbrace{1000}_{1} +
\underbrace{1000}_{2} + \underbrace{999}_{3} + \underbrace{999}_{4} +
\dots + \underbrace{2}_{1997} + \underbrace{2}_{1998} =
2\times\left(\frac{(1000)(1001)}{2}-1\right) = 1000\times1001-2 =
1000998$.

\item
For the polygon $[P_1 P_2 \dots P_n]$, choose $i$ such that a triangle
is given by $P_{i+1} P_{i+2} P_i$. Then the next triangle must border on
$\overline{P_i P_{i+2}}$ as well as an adjacent vertex, so it is either
$P_i P_{i+2} P_{i+3}$ or $P_i P_{i+2} P_{i-1}$. For each subsequent triangle, there are 2 independent choices to be made, and there are $n - 4$ choices
made in this way. There are $n$ many ways to choose the value $i$, and each
triangulation can be done in reverse for a 2-fold symmetry. Therefore the
total number of ways is given by
$$\frac{n \cdot 2^{n-4}}{2} = n \cdot 2^{n - 5}$$

\item
In the group of 5-digit numbers that don't contain a 6, we can note that
whatever the residue mod 3 of the first 4 digits, each residue mod 3 appears
exactly 3 times among the last digits. Therefore exactly a third of them are
divisible by 3:
$$\frac{(10^5 - 9^5) - (10^4 - 9^4)}{3} = \frac{37512}{3} = 12504$$
\end{enumerate}

\chapter{Combinations}

\begin{enumerate}[label={2.\arabic*}]

\item
Initially, ${10 \choose 5} = 252$ combinations can be selected. Finally,
any member of the power set except the entire set and the null set can be
selected as a combination, for a total combination count of
$$2^{10} - {10 \choose 10} - {10 \choose 0} = 1022$$
The difference is the answer, $1022 - 252 = 770$ additional combinations.

\item
If 6 games are played, all $2^6$ outcomes are equally likely, and only
${6 \choose 3}$ of them involve going to a game 7. Therefore the probability
of the Reds winning the series in game 7 is
$$\frac{{6 \choose 3}}{2^7} = \frac{5}{32}$$

N.C.: This is also a direct result of binomial probabilities, which we'll
get to soon.

For the second question, if the Reds win the series we know that they win
the final game and 3 others before it. If the Blues win $k$ games,
$0 \leq k \leq 3$, then $k+3$ games are played before the final game and
can be ordered in ${k+3 \choose 3}$ many ways. The number we're looking for is
$$\sum_{k=0}^{3} {k+3 \choose 3} = 1 + 4 + 10 + 20 = 35$$
N.C.: This sum also falls prey to the hockey-stick identity:
$$\sum_{k=0}^{3} {k+3 \choose 3} = {7 \choose 4} = 35$$
$$\sum_{k=0}^{3} {k+3 \choose k} = {7 \choose 3} = 35$$
Which suggests that there's another way to view the solution. Imagine any
series shorter than 7 games is padded out to a length of 7 games by including
wins for the losing team. This doesn't change the overall result, but gives
a bijection with the set of combinations of 7 choose 4 (or 3).

\item
(a) There are ${6 \choose 2} = 15$ chords to choose from, ${6 \choose 4} = 15$
convex quadrilaterals (each of which is uniquely described by 4 chords), and
${15 \choose 4} = 15 \cdot 91$ choices of 4 chords. Therefore the probability is
$$\frac{15}{15 \cdot 91} = \frac{1}{91}$$

(b) This question is somewhat strangely worded, and it seems to be ambiguous
whether line $AB$ intersects the interior of any of the 24 rectangles. Let us
assume that it doesn't, so that there are 12 rectangles on one side of line
$AB$ and 12 on the other. Choose one pair of rectangles that will overlap to
color black, then choose two other non-overlapping rectangles on either side.
Then there are
${12 \choose 1} \cdot 2^2 \cdot {11 \choose 2} = 2640$ ways to have exactly
one pair of black rectangles overlap. The overall probability is then
$$\frac{2640}{{24 \choose 4}} = \frac{3 \cdot 22 \cdot 40}{23 \cdot 22 \cdot 3 \cdot 7} = \frac{40}{161}$$

\item
This happens exactly when one player gets 3 odd tiles and the other two players
get 1 odd tile. There are $\frac{9!}{3!3!3!} = 1680$ ways of apportioning the
tiles, and all players get an odd sum in
$${3 \choose 1} {5 \choose 3} (2 \cdot {4 \choose 2}) = 360$$
many ways. Therefore the total probability is
$$\frac{360}{1680} = \frac{3}{14}$$

\item
If the probability of getting heads 1 out of 5 times is the same as getting
heads 2 out of 5 times,
$${5 \choose 1}p^1 (1-p)^4 = {5 \choose 2}p^2 (1-p)^3 \iff 1-p = 2p \iff p = 1/3$$
And therefore the odds of getting heads exactly 3 out of 5 times is
$${5 \choose 3}(\frac{1}{3})^3 (\frac{2}{3})^2 = \frac{40}{243}$$

\item
From any two vertices, you can build a square in two possible directions.
A quick argument that I won't rewrite here for the fact that you
can't have exactly 3 vertices in a square in the dodecagon; the fourth
vertex is necessarily in the dodecagon as well. There are $12 / 4 = 3$ of 
these such squares, and they are counted 4 times (overcounted 3 times) among
all sets of two vertices. The full count is:
$$2 \cdot {12 \choose 2} - 3 \cdot 3 = 132 - 9 = 123$$

\item
You can choose k ranks and k files on which to place the rooks, and then
there are $k!$ many ways to rearrange them, for a total of
$${m \choose k}{n \choose k}k! = \frac{m!n!}{(m-k)!(n-k)!k!} \text{ ways}$$

\item
A quick argument shows that the residues $(\text{mod }3)$ of each of these
numbers are either all the same or all different.

Case 1: $a \equiv b \pmod{3}$
$$0 \equiv a + b + c \equiv 2a + c \equiv -a + c$$
$$a \equiv b \equiv c$$
There are $2 \cdot {11 \choose 3} + {12 \choose 3} = 550$ ways for this to occur.

Case 2: $a \not\equiv b \pmod{3}$
$$0 \equiv a + b + c \not\equiv -a + c; 0 \not\equiv -b + c$$
$$a \not\equiv c; b \not\equiv c$$
There are $11 \cdot 11 \cdot 12 = 1452$ ways for this to occur, giving a total
of $550 + 1452 = 2002$ ways in which the sum of three distinct numbers from the
set can have a sum congruent to $0 \pmod{3}$.

\item
There are $2^3=8$ possibilities for the type of complementarity, of which one
is degenerate: 3 cards can't have the same shape, color, and shade.

If all 3 qualities are different, we can construct a complementary set by
starting with one of the 27 cards from the deck, then there are $2^3=8$
choices of second card, with the final card being uniquely determined.
Each of these sets can be permuted in $3!$ ways, giving a count of
$\frac{27 \cdot 8}{3!} = 36$ total sets in this case.

If two of the qualities are different, there are 3 ways to choose which of
the qualities is the same, 3 ways to choose the value of this quality, and
$\frac{9 \cdot 4}{3!} = 6$ ways to choose the elements of the subset (by
analogy with the argument in the first case above). In this case there are
a total of $3 \cdot 3 \cdot 6 = 54$ total sets.

If one of the qualities is different, there are 3 ways of choosing which of
the qualities is different, 9 ways of choosing the values of the other two
qualities, and a unique subset, giving 27 total sets in this case.

Summing these, we have $36 + 54 + 27 = 117$ complementary 3-card sets.

\item
I failed to solve this in a couple of ways, using circular counting and PIE.

This method is somewhat roundabout. Line up the remaining $m-k$ beads,
creating $m-k+1$ spaces ($m-k-1$ not including the ends).

\textbf{Direct counting}: Choose $k$ of the spaces between remaining red
marbles (not including the ends)
in ${m-k-1 \choose k}$ many ways. Then choose one of the ends in 2 ways, and
choose the other $k-1$ spots from the central spaces for a total of
$${m-k-1 \choose k} + 2{m-k-1 \choose k-1} = {m-k \choose k} +
{m-k-1 \choose k-1} \text{ ways}$$

\textbf{Overcounting}: Choose $k$ of the spaces including the ends, and
subtract out the cases when you include both ends, for a total of
$${m-k+1 \choose k} - {m-k-1 \choose k-2} \text{ ways}$$

N.C.: These can be shown to be equal through the binomial recursion or
a difference in hockey-stick identities.

\textbf{Circular counting}: I'm not as solid on this solution, but it seems
like when you take out $k$ marbles, color them blue, and replace them, there
are 2 cases.

Case 1 (red marble in position 1): There are $m-k$ spaces to fill between
red marbles, which can happen in ${m-k \choose k}$ many ways.

Case 2 (blue marble in position 1): There are $m-k-1$ remaining spaces
between the red marbles, and $k-1$ blue marbles to fill them, which can
happen in ${m-k-1 \choose k-1}$ many ways. Therefore the total is:
$${m-k \choose k} + {m-k-1 \choose k-1} \text{ ways, as in the solution
above.}$$

\item
There are 7 spaces between red beads in any case, so the skeleton of the
necklace is RBBRBBR...RBBR, with 9 places to insert the remaining $32-14=18$
blue beads. Partitioning 18 into 9 groups can be done in
${27 \choose 9} = 4686825$ many ways.

\item
First, we can notice that there are ${5 \choose 2} = 10$ lines and 
$5 \cdot {4 \choose 2} = 30$ perpendiculars
(perps), of which 6 will coincide at each of the 5 vertices and 3 will coincide
at each of the ${5 \choose 3} = 10$ orthocenters.

\textbf{Overcounting}: There are ${30 \choose 2} = 435$ pairs of perps, and
this total can be reduced in 2 ways.

Case 1: The perps are parallel, which happens in at least 3 ways per line,
of which there are 10, reducing the total by 30.

Case 2: The perps are coincident, which happens at each of the 5 vertices in
6 ways, resulting in $5 \cdot ({6 \choose 2} - 1) = 70$ overcountings, as 
well as at the 10 orthocenters, resulting in $10 \cdot ({3 \choose 2} - 1) = 20$
overcountings.

$$\text{Total intersections} = {30 \choose 2} - 3 \cdot {5 \choose 2} - 
5 \cdot ({6 \choose 2} - 1) - 10 \cdot ({3 \choose 2} - 1) = 315$$

\textbf{Constructive counting}: Consider a particular perpendicular, for
example that from $A$ to $\overline{BC}$. Excluding the vertex $A$ and the
orthocenter of $ABC$, each vertices $A, B, C, D$ have 5 perps which
intersect the given perp, for a total of 20. Counting over all perps and
adding in orthocenters and vertices, we get:
$$\frac{30 \cdot 20}{2} + {5 \choose 3} + 5 = 315 \text{ intersections}$$

\end{enumerate}

\chapter{Properties of Binomial Coefficients}

\begin{enumerate}[label={3.\arabic*}]

\item
$$\text{Assuming } m \leq k \leq n \text{,}$$
$${n \choose k}{k \choose m} = {n \choose n-k;m} = {n \choose m;k-m}
= {n \choose m}{n-m \choose k-m}$$

\item
$$(x_1 + \ldots + x_m)^n = \sum_{\substack{n_1 + \ldots + n_m = n \\ n_1, \ldots, n_m \geq 0}}
{n \choose n_1}{n-n_1 \choose n_2}{n-n_1-n_2 \choose n_3}\cdots{n_m \choose n_m}x_1^{n_1} \cdots x_m^{n_m}$$

\item
$$\sum_{i=0}^5 {10 \choose i;i}\cdot (-2)^{10-2i} = 184756$$

Alternatively, $x^2 + \frac{1}{x^2} - 2 = (\frac{(x+1)(x-1)}{x})^2$, so our desired
coefficient is that of $x^{20}$ in $(x^2 - 1)^{20}$, that is to say it is ${20 \choose 10} = 184756$.

\item
a)
$$\sum_{k=0}^n k^2 {n \choose k} = \sum_{k=0}^n 2{k \choose 2}{n \choose k} + {k \choose 1}{n \choose k}$$
$$= \sum_{k=0}^n 2{n \choose 2}{n-2 \choose k-2} + {n \choose 1}{n-1 \choose k-1}$$
$$= 2{n \choose 2}\sum_{k=0}^{n-2} {n-2 \choose k} + {n \choose 1}\sum_{k=0}^{n-1} {n-1 \choose k}$$
$$= 2^{n-2}n(n-1) + 2^{n-1}n = 2^{n-2}n(n+1)$$
b)
$$\sum_{k=0}^n \frac{(-1)^k}{k+1} {n \choose k} = \sum_{k=0}^n \frac{(-1)^k}{n+1} {n+1 \choose k+1}$$
$$= \frac{1}{n+1} \left( \sum_{k=0}^n (-1)^{k-1} {n+1 \choose k} + 1 \right) = \frac{1}{n+1}(0 + 1) = \frac{1}{n+1}$$

\item
$$\sum_{i=0}^k (-1)^i {n \choose i} = \sum_{i=0}^k (-1)^i \left({n-1 \choose i-1} + {n-1 \choose i} \right)$$
$$= \sum_{i=0}^k (-1)^i \left({n-1 \choose i} - {n-1 \choose i} \right) + (-1)^k {n-1 \choose k}$$
$$= (-1)^k {n-1 \choose k}$$

\item
This doesn't hold for $n=1$. If $n>1$ is odd, it has at least two 1s in its binary expression (first and last digits),
and we refer to 3.13a to finish the proof.

\item
The number of odds is $2^{\sigma_2 (n)}$, where $\sigma_k (n)$ is the sum of the digits of $n$ in base $k$.

\item
This only makes sense when $n \geq 2$. By symmetry we can check only ${2^n \choose i}$ where
$i < 2^{n-1}$. In these cases, $i$ written in base 2 has at least one 1 in place $k$, $k < n$. Because of this,
the addition $(2^n - i) + i$ will have at least 2 carries, in places $n$ and $k$ (carried over to become a 1 in place $n+1$).
By Theorem 3.4, this means ${2^n \choose i}$ is divisible by $2^2 = 4$.

\item
$${2p \choose p} = 2 {2p-1 \choose p-1} = 2 \cdot \frac{\prod_{\substack{n \in \mathbb{Z}_p^{\times}}} p + n}{\prod_{\substack{n \in \mathbb{Z}_p^{\times}}} n}$$
$$\equiv 2 \cdot \frac{\left( \sum_{n \in \mathbb{Z}_p^{\times}} n \right)p + 1}{1} \pmod{p^2}$$
$$\equiv 2 \cdot \left( \left( \frac{p-1}{2} \right) p^2 + 1 \right) \pmod{p^2}$$
$$\equiv 2 \pmod{p^2}$$

\item
$$\sum_{k=0}^n k {n \choose k}^2 = \sum_{k=0}^n (n-k) {n \choose k}^2 = S$$
$$2S = \sum_{k=0}^n n {n \choose k}^2 = n\sum_{k=0}^n {n \choose k}^2$$
$$= n{2n \choose n} = 2n{2n-1 \choose n-1}$$

And then, dividing by 2 gives $S = n{2n-1 \choose n-1}$.

\item
In the case of $n = 2m-1$, given $|S| = k$ we have an equal number of allowed elements of $S$ whether $k$
is even or odd. In either case, once we decide on a parity we must choose $k$ elements from $m-k$, therefore we have:
$$a_{2m-1} = 2 \sum_{k=1}^m {m-k \choose k} = 2(F_{m+1} - 1)$$
In the case of $n = 2m$, we can consider the two cases when $k$ is even and odd.
When $k$ is even, there is no difference between $S$
as a subset of $\{1, \ldots, 2m\}$ and of $\{1, \ldots, 2m+1\}$, so there are $a_{2m+1}/2$ possibilities for $S$.
Similarly when $k$ is odd, there are $a_{2m-1}/2$ possibile values of $S$. Therefore:
$$a_{2m} = \frac{a_{2m+1} + a_{2m-1}}{2} = F_{m+2} + F_{m+1} - 2 = F_{m+3} - 2$$

\end{enumerate}

\chapter{Bijections}

\begin{enumerate}[label={4.\arabic*}]

\item
Imagine a (1) at the beginning and a (6) at the end, then consider the differences between successive
rolls, which will add up to 5 across all rolls including the phantom (1) and (6). There are $6^4$ total possible
rolls, and using Theorem 4.2b, we get:
$$\frac{{9 \choose 4}}{6^4} = \frac{63}{648}$$

\item
A bijection can be established between lucky tickets and medium tickets by changing the last 3 digits
of a lucky number from $d$ to $9-d$. Then the sum of the digits will be 27, so the ticket will be medium.
The function is its own inverse, turning medium tickets into lucky tickets, and therefore is a bijection between
the sets. The sets are the same size, so $A - B = 0$.

Long way:

Ways the first 3 numbers can sum to n ($n < 10$) : ${n+2 \choose 2}$

Ways the first 3 numbers can sum to n ($10 \leq n \leq 13$) : ${n+2 \choose 2} - 3{n-8 \choose 2}$

$$ A = 2 \left( \sum_{n=0}^9 {n+2 \choose 2}^2 + \sum_{n=10}^13 \left( {n+2 \choose 2} - 3{n-8 \choose 2} \right)^2 \right) $$
$$ A = 55252 $$

$$ B = {32 \choose 5} - 6 \left( {22 \choose 5} - 3{12 \choose 5} \right) - 3{12 \choose 5} = 55252 $$

\item
For each subset of 4 points, there is only one way in which the segments can form an intersection. Conversely,
given two segments that intersect we can extract the 4 points producing them. This forms a bijection between
the set of intersections and the set of subsets of 4 points, which has size ${n \choose 4}$.

\item
In the surjective case, 49 of the quantities $a_{i+1} - a_i$ must be 1. There are 99 of these, and the set of
99-tuples forms a bijection with the set of surjective functions $f : A \to B$, so the number of these functions
is ${99 \choose 49}$.

For the non-surjective case, we will adapt the solution to problem 4.1 above.
Consider the quantities $a_{i+1} - a_i$. There are 101 of these,
they are nonnegative, and they sum to 49. The set of these 101-tuples forms a bijection with the set of
increasing functions $f : A \to B$. Therefore by Theorem 4.2b, the size of this set is ${149 \choose 49}$.

\item
First, establish a bijection between sets with 7 and sets without 7. Then, for each number less than 7,
it will be represented by addition in one set and subtraction in the other. Therefore, the sum of all numbers
other than 7 added in this way is 0. The number 7 is represented $2^6$ times, so the answer is $7 \cdot 2^6$.

\item
In binary, 2003 is represented as 11111010011. In order for $D(n) = 2$ to hold, the number must be
of the form 11...100...011...1 in binary. If the number of binary digits of $n$ is $k$ and we impose the requirement
$3 \leq k \leq 10$, then we can say that there are $k-1$ places for the change in digit to occur, and 2 of
them must be chosen for a total of ${k-1 \choose 2}$ numbers with $D(n) = 2$. Summing from $k=3$ to $10$
is easily done with the hockey stick identity for a total of ${10 \choose 3} = 120$ such numbers.

Considering 11-digit binary numbers, the greatest number of this form that is less than 2003 is
11111001111. There are 4 integers of this form with the same location for the change in the first adjacent
digit, and otherwise there are 9 places for the digits to change, giving ${9 \choose 2} = 45$ other 11-digit
numbers with $D(n) = 2$. Of these, anything with both digit changes in the last 6 places will be too big,
so we overcounted by ${6 \choose 2} = 15$. Therefore the total number of these is $120 + 4 + 45 - 15 = 154$.

Alternatively, follow the logic to get ${11 \choose 3} = 165$ numbers up to 11 binary digits. There are
$1 + 4 + 3 + 2 + 1 = 11$ that have a first digit change in the first 4 possible places, making the total
$165 - 11 = 154$.

\item
$a_{1234} = 0$ because 1234 is not expressible as the sum of powers of 3. One way to see this is by the
expression in base 3: 1200201 contains 2s and therefore doesn't show up in the given polynomial.

\item
There are $n$ one-element subsets, and for the multi-element subsets we can establish a bijection
between those containing their average and those not containing their average. This means that
$T_n - n$ is divisible by 2, or in other words that it is even.

\item
Create a bijection $f(a_1, \ldots, a_k) = (1, 1, \ldots, 1, 1 + a_1, \ldots, 1 + a_k)$ where there are $n-k$ many
ones to start the n-tuple on the RHS. This takes all partitions of $n$ to partitions of $2n$ of size $n$. For the
inverse function, ignore all 1s in the partition of $2n$ and send each $b_l > 1$ to $b_l - 1$.

\item
Each rhombus consists of one triangle in the same orientation as the grid and one inverted. Consider the vertex
of the inverted triangle, along with its orientation, to be the unique determiners of the rhombus. This creates a
bijection between the number of rhombuses in a given orientation with the number of vertices in the $(n-2)$-grid,
of which there are ${n \choose 2}$ for a total of $3{n \choose 2}$ rhombuses for the 3 possible orientations.

\item
$$\sum_{k=1}^54 6k - 2 = 8802$$

Wrong solution below, needs rework:

One way of looking at this is to have 3 numbers that correspond to how many vertices you move along each edge.
Regularly-oriented triangles would have a sum of 325, while inverted triangles would have a sum of 650. Additionally,
scalene triangles will be overcounted 5 times and equilateral triangles will be overcounted twice. This can be overcome
by requiring that $0 < a \leq b \leq c$, where these variables correspond to side "lengths" in vertices. 
In this case, you have 3 non-negative integers adding to $c-1: a-1, b-a, c-b$, where $0 \leq c-1 \leq 322$. Using Theorem 4.2a and
summing using the hockey-stick identity gives:

$$\sum_{c=1}^323 {c+1 \choose 2} = {325 \choose 3}$$

\item
The bijection is the binary $f(n) = (2n$ XOR $n) // 2$, where sequences containing 1100 or 0011 get mapped to sequences
containing 010, and sequences containing 010 are only mapped to by sequences containing 1100 or 0011. The
function takes you from a sequence of length $k + 1$ to a sequences of length $k$, and it doubly covers
the shorter sequences, giving the result:
$$b_{n+1} = 2a_n$$

\end{enumerate}

\chapter{Recursions}

\begin{enumerate}[label={5.\arabic*}]

\item
(i) Each line adds a region, and each intersection adds a region. At the nth new line added, it can intersect at
most once with each other line, for a total of $R_n = 1 + (n-1) + R_{n-1}$. Our base case is $R_0 = 1$, giving:
$$R_n = {n+1 \choose 2} + 1$$

(ii) Each new line only creates regions in the interior of intersections; that is if it intersects $(n-1)$ other lines
it will create $(n-2)$ new closed regions. Now $r_n = (n-2) + r_{n-1}$ with base case $r_2 = 0$, giving:
$$r_n = {n-2 \choose 2}$$

\item
Let $A_n$ be the number of n-digit integers starting with 1 and let $B_n$ be the number of n-digit integers
not starting with 1. Then $A_n = B_{n-1}$ and $B_n = 2(A_{n-1} + B_{n-1})$. Substitution gives
$B_n = 2(B_{n-1} + B_{n-2})$ with base cases $B_1 = 2, B_2 = 6$. Solving for the eigenvalues and using initial
values gives:
$$B_n = \frac{(1+\sqrt3)^{n+1} - (1 - \sqrt3)^{n+1}}{2\sqrt3}$$
$$A_n + B_n = \frac{(2+\sqrt3)(1+\sqrt3)^n - (2 - \sqrt3)(1 - \sqrt3)^n}{2\sqrt3}$$

\item
There are ${n-k+1 \choose k}$ $k$-element skipping subsets of $\{ 1, 2, \ldots, n\}$, and therefore there are
$\sum_{k=0}^{n} {n-k+1 \choose k} = F_{n+1}$ skipping subsets in all.

Alternatively (and more appropriately), our skipping subsets can be partition based on whether or not 
they contain $n$. If so, it cannot contain $n-1$, so removing $n$ from the subset forms a bijection with
skipping subsets of $\{1, \ldots, n-2\}$. If not, it is a skipping subset of $\{1, \ldots, n-2\}$ and the inclusion
forms a bijection. If we call $S_n$ the number of skipping subsets of a list of size $n$, we arrive at the
recursion $S_n = S_{n-1} + S_{n-2}$, and since $S_0 = F_1 = 1, S_1 = F_2 = 2$ we have $S_n = F_{n+1}$.

Alternative approach to Example 3.8:

Base case is already established, that $f_0 = F_1, f_1 = F_2$. Then by inspection of Pascal's triangle, it can
be seen that each element on the shallow diagonal is preceded by two elements on the two preceding
shallow diagonals. It follows that $f_{n+2} = f_{n+1} + f_n$ and therefore that $f_n = F_{n+1}$ generally.

\item
Thinking about the middle of the game, you will at some point move the biggest disk from its original
peg to a new peg. In this case, the new peg must be empty, and the original peg will then be empty,
meaning that the other $n-1$ disks are all on the third peg, and therefore in the correct order. This
means that you have solved exactly the $(n-1)^{\text{th}}$ Tower of Hanoi game up to this point. By
the same logic, you will solve the $(n-1)^{\text{th}}$ game again to put all the disks back onto the
largest disk. If we call the number of moves to solve the $n^{\text{th}}$ game $H_n$, this gives us the
recursion $H_n = 2H_{n-1}+1$. The base case $H_1 = 1$ propagates out to the solution $H_n = 2^n - 1$.

\item
As we know from Problem 5.3, there are $F_{11} = 144$ ways for heads not to occur on consecutive
tosses, therefore the probability is $P = \frac{144}{2^{10}} = \frac{9}{64}$.

\item
If we call $P(n, k)$ the probability of making $k$ baskets in $n$ attempts, we have the following recursion
from Bayes': $P(n, k) = P(n-1, k-1)\frac{k-1}{n-1} + P(n-1, k)\frac{n-k-1}{n-1}$. Ignoring the probabilities on
the RHS, the fractions add to $\frac{n-2}{n-1}$, suggesting how the steady state can evolve. We clearly have
a steady state for low $n, k$: $P(3, 1) = P(3, 2) = \frac{1}{2}, P(4, 1) = P(4, 2) = P(4, 3) = \frac{1}{3}$, and
assuming that $P(n, k) = \frac{1}{n-1}$ for all $k$ (base case already established), we get:

$$P(n+1, k) = \frac{k-1}{n(n-1)} + \frac{n-k}{n(n-1)} = \frac{1}{n}$$

Therefore $P(100, 50) = \frac{1}{99}$.

\item
I actually get different values for these two, both in their recursion and in their closed forms. For example,
when $n=3$ we have every permutation except for $(3 \;\; 2 \;\; 1)$ belonging to $A$, but both permutations
beginning with $2$ are not in $B$, so $|A_3| = 5 \neq 4 = |B_3|$.

For $A_n$, permutations are in this if, when read from left to right, the numbers decrease at most once. There
is only one permutation for which the numbers decrease 0 times, so we will treat this as a unique case. For the
other $|A_n| - 1$ many of them, there are two places we can place $(n+1)$ such that the property still holds, and
they are just left of the decrease or at the right end. These are two different spots here, so this case represents
$2|A_n| - 2$ permutations in $A_{n+1}$. For the unique case of the monotonically increasing permutation, the
new value can be placed anywhere, in $n+1$ ways, to form a member of $A_{n+1}$. Therefore we have:

$$|A_{n+1}| = 2|A_n| + n - 1$$

Using the base case $A_1 = 1$ and quickly checking a linear combination of a linear polynomial and an exponential,
we get $A_n = 2^n - n$ as the general non-recursive formula.

If the ``or'' is meant to be an \emph{exclusive}
or, the permutations belonging to $A$ can't start with 1, which changes the situation. In this case, each $a \in A$
starts with an increasing sequence, followed by 1 and then the rest of the numbers in increasing order. The
number of ways it is possible to have an increasing nonempty subset of $\{2, \ldots, n\}$ is $2^{n-1} - 1$, so in
this case we have ${A'}_n = 2^{n-1} - 1$.

The recursion is similar to the case above, except that the monotonic
increasing case doesn't exist here, so we have simply ${A'}_{n+1} = 2{A'}_{n} + 1$, where the ``+1'' term comes
from the permutation $((n+1) \;\; 1 \;\; 2 \;\; \ldots \;\; n)$.

For $B_n$, consider the leftmost end. If it is not 1 or $n$, call it $i$. Both $i+1$ or $i-1$ will appear to the right,
and they each only have one possible value which must be to its right, up to $n$ and down to 1. One of these two
sequences may terminate on the right-hand side of the permutation, but that leaves the other sequence to end
with $2 \ldots 1 \ldots$ or $n-1 \ldots n \ldots$, a contradiction to its membership in $B_n$. If 1 or $n$ is chosen,
the rest of the sequence is a member of $B_{n-1}$, so we have the recursion:

$$|B_{n+1}| = 2|B_n|$$

Using the base case $B_1 = 1$ gives $B_n = 2^{n-1}$ as the general non-recursive formula.

Finishing the logic in Example 5.8, the ``good'' permutations of $\{ n, n-1, \ldots, n-k+1\}$ are identical to the
``good'' permutations of $\{ k, k-1, \ldots, 1\}$ represented in $B_k$, which has size $2^{n-1}$. As $k$ can be
any positive integer up to $n$, we have $b_n = \sum_{k=1}^n 2^{k-1} = 2^n - 2$.

\item
This is the recursion behind the subfactorial function, also known as the problem of derangements.
The idea is to permute n objects such that no object ends where it started. To see that this is satisfied
by the recursion, consider the $(n+1)^{\text{th}}$ object and the $n$ cases where it ends up in position
$i, 1 \leq i \leq n$. Then if object $i$ ends up in position $n+1$, this reduces to a derangement of $n-1$
objects. If not, there are $n$ objects going to $n$ places, none of them remaining still -- this is a
derangement of $n$ objects. Therefore $D_{n+1} = n(D_n + D_{n-1})$.

Based on the given base
cases: $D_1 = !1 = 0, D_2 = !2 = 1$, we can conclude that $D_n = !n = \left[\frac{n!}{e}\right]$ (one of my
favorite formulas!).

\item
Call the number of $n$-letter ARMLovian words that start with a vowel $V_n$ and those that start
with a consonant $C_n$.

Case 1a: Starts with vowel + consonant + vowel. In this case, removing the first vowel turns it into a
$(n-1)$-digit word, so there are $C_{n-1}$ of these.

Case 1b: Starts with vowel + consonant + consonant + vowel. In this case, removing the first two
vowels forms a bijection with a set with size $C_{n-2}$.

Case 1c: Starts with vowel + vowel. In this case, removing the first vowel makes it an word of length
$n-1$ starting with a vowel, of which there are $V_{n-1}$.

Overall from Case 1 we get $V_n = V_{n-1} + C_{n-1} + C_{n-2}$.

Case 2: Starts with a consonant. In this case, removing that consonant gives a word starting with
a vowel: $C_n = 3V_{n-1}$.

Combining these gives $V_n = V_{n-1} + 3V_{n-2} + 3V_{n-3}$, with the same relation existing for $C_n$. This is
an easy recurrence to calculate 7 times, and I'm too lazy to do a 3x3 eigenvalue decomposition with a 3-equation
system. Base cases $C_1 = 0, C_2 = 3, C_3 = 12$ give $C_n = \{ 21, 66, 165, 426, 1119 \}$
for $n = \{4, 5, 6, 7, 8\}$:

$$V_7 + C_7 = 373 + 426 = 799$$

Alternatively you could use the same recursion for the total number of words $W_n: W_1 = 1, W_2 = 7, W_3 = 19$
to calculate out to $W_7 = 799$.

\item
From the same argument as the previous problem, $S_n = S_{n-1} + S_{n-3} + S_{n-4}$. The base case is:
$S_1 = 1, S_2 = 1, S_3 = 2, S_4 = 4$. Computing out the next few values gives 6, 9, 15, 25, 40, 64, 104, 169,
273, 441, 714, 1156.

From this we see that $S_{2k} = F_k^2$ for $k \leq 5$, from which we get
$$S_{4k+1} = \sum_{i=1}^{k} S_{2i} + 1 = F_{2k}F_{2k+1}$$
$$S_{4k+3} = \sum_{i=2}^{k+1} S_{2i} + 2 = F_{2k+1}F_{2k+2}$$

Generally, $S_{2k+1} = F_{k}F_{k+1}$

Which in turn, implies:
$$S_{2k+2} = F_{k}F_{k+1} + F_{k}F_{k-1} + F_{k-1}^2$$
$$=F_{k}F_{k+1} + F_{k-1}(F_{k} + F_{k-1})$$
$$=F_{k}F_{k+1} + F_{k+1}F_{k-1}$$
$$=F_{k+1}(F_k + F_{k-1})$$
$$=F_{k+1}^2$$

And by induction, we have proved these relations for all k, and we are finished.

\item
Let $a_n$ denote the number of such colorings whose right end is not colored completely red, and $b_n$
the number of colorings whose right end is completely red. For the first set, there are 3 ways to color the right
end -- RB, BR, BB -- so we establish the recursion:

$$a_{n+1} = 3(a_n + b_n)$$
$$b_{n+1} = a_n$$

Combining these, we are armed with the recursion $a_{n+1} = 3(a_n + a_{n-1})$, and because $\{b_n\}$ is
equal to an index-shifted $\{a_n\}$, it also satisfies the recursion, as does any linear combination of the two.
In particular, $c_n = a_n + b_n$ also satisfies this recursion. It is easy to establish the base case of $c_0 = 1, c_1 = 4$.
Then, using the notation $3^t || n$ to mean that the prime factorization of $n$ contains $3^t$ (and $3^{t+1}$
doesn't divide $n$), we can conjecture that $3^k || c_{2k}, c_{2k+1}$ for all $k$.

With the base case established, assume it is true for all $k < 2n$. Then
$$3^{n-1} || (c_{2n-1} + c_{2n-2})$$
$$3^n || 3(c_{2n-1} + c_{2n-2}) = c_{2n}$$

From which we can deduce by the same logic that $3^k || c_{2n+1}$ and finish our strong induction. We have
proved that $3^{1000}$ is the greatest power of 3 that divides $c_{2001}$.

\item
Exeter is an easier case, where there are ${n \choose r}$ ways to choose the girls and the boys, then there
are $r!$ ways to pair them together, for a total of $E(n, r) = {n \choose r}^2 r! = {n \choose r}\frac{n!}{(n-r)!}$
possibilities. This is the chapter on recursions, so we should establish both in case one is easier to compute
for Lincoln than another:

$$E(n+1, r) = \left(\frac{n+1}{n+1-r} \right)^2 E(n, r); \text{ base case: } E(r, r) = r!$$
$$E(n, r+1) = \frac{(n-r)^2}{r+1} E(n, r); \text{ base case: } E(n, 1) = n^2$$

In Lincoln, $g_1$ can only choose $b_1$, then $g_2$ has 2 choices, and in general $g_k$ has $k$ choices if
all of the previous girls have already chosen. As the choices are independent, we can multiply them together
to reach the same first base case: $L(r, r) = \prod_{k=1}^r k = r!$. Alternatively, if only one girl is making a choice,
then $g_k$ has $2k-1$ choices, and we get the same second base case: $L(n, 1) = \sum_{k=1}^n 2k-1 = n^2$.

If $r$ girls have already made their choice, then there are 2 cases when another girl makes her choice. Choosing
$g_n$ allows you to choose for her last, giving $(2n - 1 - r)L(n, r)$ possibilities. Not choosing $g_n$ is
equivalent to a choice in $L(n-1, r+1)$, so we have:

\begin{align}
L(n, r+1) = (2n-1-r)L(n-1, r) + L(n-1, r+1)
\end{align}
$$=(2n-1-r)L(n-1, r) + (2n-3-r)L(n-2, r) + L(n-2, r+1)$$
$$=\sum_{k=1}^n (2n - 2k + 1 - r)L(n-k, r)$$

At this point, rather than trying to manipulate this recursion to get a simpler closed form, we can just check
if $E(n, r)$ satisfies the recurrence (5.0.1):

\begin{align*}
& (2n-1-r)E(n-1, r) + E(n-1, r+1) \\
& =(2n-1-r)\frac{((n-1)!)^2}{r!((n-1-r)!)^2} + \frac{((n-1)!)^2}{(r+1)!((n-2-r)!)^2} \\
& =\frac{((n-1)!)^2}{r!((n-2-r)!)^2}\left(\frac{2n-1-r}{(n-1-r)^2} + \frac{1}{r+1}\right) \\
& =\frac{((n-1)!)^2}{r!((n-2-r)!)^2}\left(\frac{2nr + 2n - r^2 - 2r - 1 + n^2 - 2nr - 2n + r^2 + 2r + 1}{(r+1)(n-1-r)^2} \right) \\
& =\frac{((n-1)!)^2}{r!((n-2-r)!)^2}\left(\frac{n^2}{(r+1)(n-1-r)^2} \right) \\
& =\frac{(n!)^2}{(r+1)!((n-1-r)!)^2} \\
& =E(n, r+1) \text{, as desired.}
\end{align*}

As both $E(n, r)$ and $L(n, r)$ satisfy the same base case and same recursion, they must be equal.

\end{enumerate}

\chapter{Inclusion and Exclusion}

\begin{enumerate}[label={6.\arabic*}]

\item
\begin{align*}
& |A_3 \cap A_4 \cap \overline{A_5}| = 2001 - (|A_3| + |A_4| + |\overline{A_5}|) + (|A_{12}| + \left\lfloor\frac{4}{5}(|A_3| + |A_4|)\right\rfloor) \\
& = 2001 - 667 - 500 - 1601 + 166 + 533 + 400 = 332
\end{align*}

\item
Let $S_r, S_g, S_b$ denote the sets of arrangements in which red, green, and blue marbles respectively
are all together. Then by PIE, we have:

\begin{align*}
& |\overline{S_r} \cap \overline{S_g} \cap \overline{S_b}| = |S| - |S_r| - |S_g| - |S_b| + |\overline{S_r} \cap \overline{S_g}| + |\overline{S_g} \cap \overline{S_b}| + |\overline{S_b} \cap \overline{S_r}| \\
& = {12 \choose 3;4;5} - {10 \choose 1;4;5} - {9 \choose 1;3;5} - {8 \choose 1;3;4} + {7 \choose 1;1;5} + {6 \choose 1;1;4} + {5 \choose 1;1;3} \\
& = 27720 - 1260 - 504 - 280 + 42 + 30 + 20 = 25768
\end{align*}

\item
From PIE, we can compute these directly. For $O_n$ -- the number of permutations of
$\{1, 2, \ldots, n\}$ with exactly one fixed point -- we have:

\begin{align*}
& O_n = n!\sum_{k=1}^n \frac{(-1)^{k-1}}{(k-1)!} \\
& = n!\sum_{k=0}^{n-1}\frac{(-1)^k}{k!} \\
& = n!\sum_{k=0}^{n}\frac{(-1)^k}{k!} - (-1)^n\\
& = D_n - (-1)^n \text{, as desired.}
\end{align*}

Instead, if we use a bijection, we can observe that any permutation fixing exactly 1 point is a
derangement of the other $n-1$ points, so we have:

\begin{align*}
& O_n = nD_{n-1} \\
& = n!\sum_{i=0}^{n-1}\frac{(-1)^i}{i!}
\end{align*}

And the proof finishes as the equation above.

\item
There are ${2n \choose 2} = n(2n-1) \geq n^2 + n > n^2 + 1$ many segments to choose from, and there
are ${2n \choose 3} = \frac{2n(2n-1)(n-1)}{3}$ many possible triangles to form. We will prove the statement
by induction on $n$. The base case of $n=2$ features 4 points and 6 segments, of which you are choosing 5.
WLOG let $\overline{P_1 P_2}$ be the segment that is not chosen; then $\Delta P_1 P_3 P_4, \; \Delta P_2 P_3 P_4$
are both triangles formed by the chosen segments.

Next, we assume the statement is true for $n=k$ and examine the case of $n=k+1$. Notice that any point forms
at most $k$ segments with other points, and therefore removing one point, as well as all its attached segments,
would still leave at least $k^2 + k + 2 \geq k^2 + 1$ segments among $k$ points. From here, we apply the induction
to show the existence of a triangle in the case of $n=k+1$.

\item
This is equivalent to Theorem 6.1 and the commutative and associative properties of addition.

\item
Possibly needs some work to put in terms of the totient function, since that seems sufficient to solve the
problem (though I haven't figured out how).

Any factor $f$ of $n^2$ smaller than $n$ can be paired with a factor larger than $n$: $f \cdot \frac{n^2}{f} = n^2$,
therefore half of one less than the number of factors of $n^2$ are less than $n$. Writing the prime factorization
$n = \prod_{i=1}^k p_i^{n_i}$, we have $n^2 = \prod p_i^{2n_i}$. There are

$$ \prod (n_i + 1) - 1 = \sum_{j=1}^k \sigma_j \text{ factors of $n$ less than $n$,}$$

where $\sigma_j = \sum_{1 \leq n_{i_1} < \ldots < n_{i_j} \leq k} \prod_{\ell = 0}^j n_{i_{\ell}}$ is the $j^{\text{th}}$
cyclic sum. Similarly, there are

$$ \frac{ \prod (2n_i + 1) - 1}{2} = \sum_{j=1}^k 2^{j-1}\sigma_j \text{ factors of $n^2$ less than $n$.}$$

Taking the difference, we get $\sum_{j=2}^k (2^{j-1}-1)\sigma_j$ divisors of $n^2$ that are less than $n$ and
do not divide $n$.

\item
We're first going to calculate the least common multiple of 3 numbers. We can use the
well-established base case of

\begin{align*}
& [a, b] = \frac{ab}{(a, b)}
\end{align*}

Continuing, we have:

\begin{align*}
& [a, b, c] = [[a, b], [b, c]] \\
& = \frac{[a, b][b, c]}{([a, b], [b, c])} \\
& = \frac{ \left( \frac{(ab^2c)}{(a, b)(b, c)}\right) }{\left( \frac{b(a, c)}{(a, b, c)} \right)} \\
& = \frac{a b c (a, b, c)}{(a, b)(b, c)(c, a)} \text{. Rearranging this gives} \\
& \frac{[a, b, c]}{(a, b, c)} = \frac{abc}{(a, b)(b, c)(c, a)}
\end{align*}

Squaring both sides, we get:

\begin{align*}
& \frac{[a, b, c]^2}{(a, b, c)^2} = \frac{a^2 b^2 c^2}{(a, b)^2 (b, c)^2 (c, a)^2} \\
& = \frac{\left( \frac{ab}{(a, b)} \right) \left( \frac{bc}{(b, c)} \right) \left( \frac{ca}{(c, a)} \right) }{(a, b)(b, c)(c, a)}\\
& = \frac{[a, b][b, c][c, a]}{(a, b)(b, c)(c, a)} \text{, as desired.}
\end{align*}

\item
We'll start by creating a bijection. The inequality $|a| + |b| \leq |c|$ is the triangle inequality, so instead of distributing
tickets we'll be building triangles with integer side lengths $a, b, c$ and perimiter $a+b+c = 61$. Because 61 is odd,
there are no degenerate triangles with $a=0$ or $a + b = c$.

This is the chapter on PIE, so while this can be solved with direct counting and recursion, we'll use the formula

$$|A \cap B \cap C| = |A \cup B \cup C| - |A| - |B| - |C| + |A \cap B| + |B \cap C| + |C \cap A|$$

$A$ is the number of ways we can distribute 61 tickets to the 3 dorms, given that we distribute between 1 and 30
to the first dorm. $B$ and $C$ are defined similarly for the second and third dorms, respectively. By the Pigeonhole
Principle, one of the sides must be between 1 and 30, except in the 3 cases where two sides are zero. This allows us
to calculate $|A \cup B \cup C|$ as the number of ways to sum 3 nonnegative integers to 61, minus 3. From there,
$A$ occurs when the first term is not zero or above 30; complementary counting gives the $|A|, |B|, |C|$ terms.
Similarly, $A \cap B$ occurs when all 3 side are positive, except for when $a > 30$ or $b > 30$, which
can't happen simultaneously. Therefore we have:

\begin{align*}
& |A \cap B \cap C| = ({63 \choose 2} - 3) - 3({63 \choose 2} - {62 \choose 1} - {32 \choose 2}) + 3({60 \choose 2} - 2{30 \choose 2}) \\
& = 1950 - 4185 + 2700 = 465 \text{ ways to distribute the tickets.}
\end{align*}

Checking our answer with direct counting, given $a$ we have $c = 61 - a - b \leq 30 \iff b \geq 31 - a$, so we can pick
$b$ to be anything between $(31-a)$ and 30, inclusive. This gives us $a$ choices, and so we have:

$$ \sum_{a=1}^{30} a = 465 \text{, as expected.}$$

\item
For the purposes of PIE, let's first calculate how many ways 3 sets can satisfy both properties. If $A, B \subseteq C$
and $A \subseteq B, C$, then $A \subseteq B \subseteq C$. Given $A$ has $a$ elements (and $b, c$ are defined
similarly), there are ${n \choose c}$ many subset choices for $C$, ${c \choose b}$ many subset choices
for $B$, and ${b \choose a}$ subset choices for $A$ for a total of:

\begin{align*}
& \sum_{0 \leq a \leq b \leq c \leq n} {n \choose a;b;c} = \sum_{0 \leq a \leq b \leq c \leq n} {n \choose c}{c \choose b}{b \choose a} \\
& = \sum_{0 \leq b \leq c \leq n}{n \choose c}{c \choose b}\sum_{a=0}^b {b \choose a} \\
& = \sum_{c=0}^n{n \choose c}\sum_{b=0}^c{c \choose b}2^b \\
& = \sum_{c=0}^n{n \choose c}3^c = 4^n \text{ ways to have } A \subseteq B \subseteq C \subseteq \{1, 2, \ldots, n \}
\end{align*}

Compare this to the number of ways to choose $A, B, C \subseteq \{1, 2, \ldots, n \}$, which is $(2^n)^3 = 8^n$.

Given a $c$-element set $C$, there are $2^c$ ways to choose a subset, and therefore $4^c$ ways to choose 2 subsets.
Summing, we get: $\sum_{c=0}^n 4^c = \frac{4^{n+1} - 1}{3}$.

We create a bijection from this case to the case where one of the sets is a subset of the other two, because
\begin{align*}
A \subseteq B, C \iff \overline{B}, \overline{C} \subseteq \overline{A}
\end{align*}

Therefore the total number of ways in which this can be achieved \emph{inclusively} (that is, $A \subseteq B \subseteq C$
is allowed) is:

\begin{align*}
& |A_1 \cup A_2| = |A_1| + |A_2| - |A_1 \cap A_2| \\
& = \frac{2(4^{n+1} - 1)}{3} - 4^n = \frac{5\cdot 4^n - 2}{3}
\end{align*}

\emph{Exclusively}, this would be $\frac{2(4^n-1)}{3}$ instead.

\item
Of interest, we can establish a bijection $f : \mathcal{P}(S)^k \to \mathcal{P}(S)^k$ by \\ 
$f(S_1, S_2, \ldots, S_k) = (\overline{S_1}, \overline{S_2}, \ldots, \overline{S_k})$, with proof of bijectivity from \\
$f = f^{-1}$. Then $$\cap S_i = \emptyset \iff \cup \overline{S_i} = S$$

This isn't necessary, but provides some motivation; it's clearer to see that $\cup \overline{S_i} = S$ if and only if for each
$s \in S$ there is some $j, \;1 \leq j \leq k$ such that $s \in \overline{S_j} \iff s \not\in S_j$. There are $2^k - 1$ ways to
choose membership of $s$ in order to satisfy this, and all choices of $|S|$ values of $s$ are independent, so we can
multiply the possibilities to get $(2^k - 1)^{|S|}$ ways in all.

\item
The analogy to PIE here is that lcm $\to \cup$, gcd $\to \cap$, $+ \to \times$, $- \to \div$, and so

\begin{align*}
[a_1, \ldots, a_n] \geq \frac{a_1a_2\cdots a_n}{\prod_{1 \leq i < j \leq n}(a_i, a_j)} \to |\cup A_i| \geq \sum|A_i| - \sum|A_i \cap A_j|
\end{align*}

This is one of the Bonferroni inequalities and has been established earlier in the chapter. To see more explicitly why
this is true, consider a prime factor $p$ of $[a_1, \ldots, a_n]$, and WLOG let it appear $k$ times as a factor of the
$\{ a_i \}$. Then it will appear once on the LHS and $k - {k \choose 2} = \frac{3k - k^2}{2} \leq 1$ times on the RHS,
with equality only if $k=1, 2$.

\item
WLOG let $x_2 > x_1$, and let $\Delta_{1,2} = 1 - \frac{x_1}{x_2} \in (0, 1)$. Clearly if $\Delta_{1,2}$ is rational, we can write
$px_1 = qx_2$ so that $\lfloor npx_1 \rfloor = \lfloor nqx_2 \rfloor$ for all $n \in \mathbb{Z}_+$, and
$|S(x_1) \cap S(x_2)|$ is not finite. Assume then that $|S(x_1) \cap S(x_2)|$ is finite, so that $\Delta_{1,2}$ is
irrational and $\exists N : \lfloor nx_1 \rfloor \neq \lfloor kx_2 \rfloor \;\; \forall k \in \mathbb{Z}_+, n > N$.
Choose an approximant $p/q$ from the continued fraction representation of $\Delta_{1,2}$ such that
$q > N\cdot \frac{p}{q} \Delta_{1,2}$, then $qx_1$ and $(q-p)x_2$ differ by a tiny amount, meaning their floors
will almost certainly be equal, contradicting the assumption. \\

N.B. 3 real numbers wasn't necessary here, nor was the condition that $\frac{1}{x_1} + \frac{1}{x_2} + \frac{1}{x_3} > 1$.

\end{enumerate}

\chapter{Calculating in Two Ways: Fubini's Principle}

\begin{enumerate}[label={7.\arabic*}]

\item
(a) This is obviously a mistake, and I believe it was meant to say ``five-digit numbers'', in which case the final digit must
be 5. The other digits can be rearranged in any way, in a total of $4! = 24$ ways. The digits sum to 20, and each digit is
used 6 times in each place, for a total sum of $$20\cdot 6 \cdot 11110 + 24\cdot 5 = 1599960$$

(b)

\begin{align*}
& \prod_{i=1}^{59} (24^{\frac{1}{i+1} + \ldots + \frac{1}{60}})^i = \prod_{i=1}^{59} 24^{\frac{i}{i+1} + \ldots + \frac{i}{60}} \\
& = \prod_{i=1}^{59} \prod_{j=i+1}^{60} 24^{\frac{i}{j}} \\
& = \prod_{j=2}^{60} \prod_{i=1}^{j-1} 24^{\frac{i}{j}} \text{ (Fubini's Principle)} \\
& = \prod_{j=2}^{60} (24^{\frac{1}{j}})^{\sum_{i=1}^{j-1} i} \\
& = \prod_{j=2}^{60} (24^{\frac{j-1}{2}}) \\
& = 24^{\sum_{j=2}^{60} \frac{j-1}{2}} = 24^{885} = 2^{2655}\cdot 3^{885} \\ \\
& 2655 + 885 = 3540 \text{, our final answer.}
\end{align*}

\item
Let S be the set of ordered pairs $(i, s)$, where $i$ is the student number between 1 and 60 and $s$ is any positive
integer less than or equal to the student's
AIME score. Then $a_i = |S(i, *)|$ and $b_s = |S(*, i)|$, so the statement we are asked to prove is Fubini's Principle,
already proven in the chapter:

$$\sum |S(i, *)| = \sum |S(*, s)|$$

\item
If we define $t$ as the total number of triangular faces and $p$ as the total number of pentagonal faces, we have
$t = VT/3, p = VP/5$ and therefore $t + p = V\left( \frac{T}{3} + \frac{P}{5} \right) = 32 \iff V(5T + 3P) = 480$. 
Counting edges, we have $\frac{V(T + P)}{2}$ in total. Then from Euler's Formula, we can count in a different way to
get

$$F - E + V = 2 \iff \frac{V(T + P)}{2} = 30 + V \iff V(T + P - 2) = 60$$

Finally, in order to make a corner we must have

$$108P + 60T < 360 \implies (T, P) \in \{ (1, 2), (2, 1), (2, 2), (3, 2) \}$$

Checking these with the two equations, the only
solution is $V=30, T=P=2$, giving $200 + 20 + 30 = 250$ as the final answer.

\item
Count the number of committees in 2 ways, first as a choice of senators: ${30 \choose 3}$. Next, let the number
of all-enemy committees be E, and all-friend committees be F. We can construct the number of 2-enemy-pair
committees by choosing a senator and choosing two enemies, though this triply counts the all-enemy committees :
$30{6 \choose 2}-3E$. Similarly, 1-enemy-committees are $30{23 \choose 2} - 3F$, and adding we get

\begin{align*}
& {30 \choose 3} = E + (30{6 \choose 2}-3E) + 30{23 \choose 2} - 3F + F \\
& E + F = \frac{30\left( {6 \choose 2} + {23 \choose 2} \right) - {30 \choose 3}}{2} \\
& = 15(15 + 253) - 2030 = 1990
\end{align*}

Alternative 1: Complementary counting.
Count the number of mixed committees by choosing a senator, choosing one of their friends and one enemy.
There is one other senator in the committee who has both a friend and an enemy, so this double counts.
Then we have:

$$E + F = {30 \choose 3} - \frac{30\cdot 23 \cdot 6}{2} = 1990$$

\item
First, let's find $\sum_{\pi \in S_n} |k - \pi(k)|$ for a fixed $k$. There are $(n-1)!$ permutations that take $k$ to $l$,
for any $l \in \{ 1, 2, \ldots, n \}$. Therefore

\begin{align*}
& \sum_{\pi \in S_n} |k - \pi(k)| = (n-1)!\left( (k-1) + (k-2) + \ldots + 1 + 0 + 1 + \ldots + (n-k) \right) \\
& = (n-1)!\left( {k \choose 2} + {n - k  +1 \choose 2} \right)
\end{align*}

Now we can use Fubini's Principle to calculate:

\begin{align*}
& \sum_{\pi \in S_n} f(\pi) = \sum_{\pi \in S_n} \sum_{k=1}^n |k - \pi(k)| \\
& = \sum_{k=1}^n \sum_{\pi \in S_n} |k - \pi(k)| \\
& = \sum_{k=1}^n (n-1)!\left( {k \choose 2} + {n - k  +1 \choose 2} \right) \\
& = (n-1)! \left( {n+1 \choose 3} + {n+1 \choose 3} \right) \\
& = \frac{2(n+1)n(n-1)(n-1)!}{3\cdot 2} = \frac{(n-1)(n+1)!}{3}
\end{align*}

\item
If every subset already has at least 3 elements, then with 3 elements in each subset there are already
$7{3 \choose 2} = 21$ pairs in the subsets. There are ${7 \choose 2} = 21$ pairs overall, so these subsets
are maximal. Clearly two subsets may not share more than one element, so we will prove that they may
not be disjoint either. To establish a contradiction, WLOG let $M = \{ 1, 2, 3, 4, 5, 6, 7 \}; \; A_1 = \{ 1, 2, 3 \}; \;
A_2 = \{ 4, 5, 6 \}$. Then $A_3, \ldots, A_7$ must all include one element from $A_1$, one element from $A_2$, and 7;
otherwise they will include a pair from either $A_1$ or $A_2$. However, from the Pigeonhole Principle there are
3 elements to take from either of these sets, and 5 sets which do so, meaning that there will be a pair of subsets
in $\{ A_3, \ldots, A_7 \}$ that contain the same pair of elements. This is a contradiction, so two subsets cannot
be disjoint and therefore must share exactly one element.

\item
I don't understand how Fubini's Principle applies here, but this seems like a nice induction. In the combinatorial
model of having $k$ red balls and $n-k$ blue balls, there are ${k \choose 2}$ ways of picking two red balls,
$k(n-k)$ ways of picking one of each kind, and ${n-k \choose 2}$ ways of picking two blue balls. This gives us
the identity ${n \choose 2} = k(n-k) + {k \choose 2} + {n-k \choose 2}$.

The hypothesis is that $f(n) = {n \choose 2}$, and this is true for the base case $f(1) = 0$, or $f(2) = 1$ if you're
more comfortable with that. Under the strong inductive hypothesis that $f(k) = {k \choose 2}$ for all $k < n$,
we have $f(n) = k(n-k) + {k \choose 2} + {n-k \choose 2} = {n \choose 2}$ and the proof is complete.

\item
Direct method: Let $S = \{ \left( (s_i, s_j), p_k \right) : s_i \text{ or } s_j \text{ correctly solved } p_k \}$.
For a given $p_k$, at most 80 students failed to solve it, so $|S(*, p_k)| \geq {200 \choose 2} - {80 \choose 2}$, and

\begin{align*}
& \sum_{1 \leq i < j \leq 200} |S \left( (s_i, s_j), * \right)| = \sum_{k=1}^6 |S(*, p_k)| \\
& \geq 6\left( {200 \choose 2} - {80 \choose 2} \right) = 6{200 \choose 2} - 18960
\end{align*}

To achieve a contradiction, assume that $|S\left( (s_i, s_j), * \right)| \leq 5$ for all $i, j$. Then:

\begin{align*}
\sum_{1 \leq i < j \leq 200} |S \left( (s_i, s_j), * \right)| \leq 5{200 \choose 2} = 6{200 \choose 2} - 19900
\end{align*}

In short, $\sum |S| \leq 6{200 \choose 2} - 19900 < 6{200 \choose 2} - 18960 \leq \sum |S|$, a contradiction.
Therefore there must be one pair that
solved every problem correctly.

Alternative: If any participant scores 5 or 6, it is clear that they can be paired with someone in order to have
solved all of the problems. If no participant scores a 5 or a 6, at least $120\cdot 6 - 200 \cdot 3 = 120$ participants
must have scored a 4. Take one of them, say $s_i$ and let

$$I = \{ (s_j, p_k ) : s_j \text{ correctly solved } p_k \text{, where } p_k \text{ was not correctly solved by } s_i \}$$
$$\sum_{j \neq i} |I(s_j, *)| = \sum_{k=1}^6 |I(*, p_k)| \geq 2 \cdot 120 = 240$$

By the Pigeonhole Principle, at least one of the 199 other students -- say $s_j$ -- must have $|I(s_j, *)| = 2$, and
then we know that $(s_i, s_j)$ is a pair that correctly solved all 6 problems.

\item
Each teacher must spend at least one hour giving their presentation and 10 hours watching the other teachers'
presentations. This makes the minimum possible time 11 hours, and if we have only teacher present at a time,
this gives us the minimal 11-hour conference.

\item
A particular row contains $\frac{100 \cdot 75}{2} = 3750$ pairs of differing color, so there are 375000 of these
row pairs in all rows. There are ${100 \choose 2} = 4950$ pairs of columns, and by $375000 = 4950\cdot75 + 3750$
there are at least 3750 pairs of columns containing at least 76 pairs of differing color in a row. Choose one of these
column pairs, and limit the grid to $76 \times 2$ differing row color pairs in these two columns. By the problem
assumptions, there are at most 25 of each color in each of the two grid columns. Due to this, and $76 = 100 - 24$,
each color must appear at least $25 - 24 = 1$ time in each column of the grid.

Suppose, in order to achieve a contradiction, that there are not 4 points making a rectangle in this grid such that
all 4 points are different colors. Because each color must appear at least once in the columns, the row pairs must
all belong to one of the following sets: $\{ (r, g), (r, b), (r, y) \}, \{ (r, g), (g, b), (g, y) \}, \{ (r, b), (g, b), (b, y) \}, \{ (r, y), (g, y), (b, y) \}$.
Each of these sets contains a color that appears in each pair; therefore there is a color which appears in all 76
rows of the grid. By the Pigeonhole Principle, these 76 points sorted into 2 columns means there is a column with
at least 38 points of the same color, a contradiction to the requirement that no color appears more than 25 times
in any column.

N.B. You always have at least 73 4-color groups show up in a given grid, e.g. 25 pairs of (r, g), 24 of (r, b), 25 of (g, b), 1 of (r, y),
and 2 of (g, y). 
In addition, there is no overlap between the different column pairs, so this actually proves the existence of at least
$3750 \cdot 73 = 273750$ such 4-color rectangles.

\item
There are ${b \choose 2} = b \cdot \frac{b-1}{2}$ pairs of judges in all. Let $S = \{ c, (j_1, j_2) : c \text{ is a contestant and }
j_1, j_2 \text{ are a pair of judges that agree on } c \}$. Before we proceed on counting $|S|$ in two ways, let's examine
how we can bound $|S(c, *)|$. If a contestant receives $n$ passes and $b-n$ fails, there are $n(b-n)$ pairs of judges in
disagreement. This is minimized when $n = \frac{b \pm 1}{2}$, when $n(b-n) = \frac{b^2 - 1}{4}$, and therefore at
least $\frac{b^2 - b}{2} - \frac{b^2 - 1}{4} = \frac{b^2 - 2b + 1}{4} = \left( \frac{b-1}{2} \right)^2$ pairs of judges agree.

\begin{align*}
a\left(\frac{b-1}{2}\right)^2 \leq \sum |S(*, (j_1, & \, j_2))| = \sum|S(c, *)| \leq \frac{b(b-1)}{2}k \\
bk & \geq \frac{a(b-1)}{2} \\
\frac{k}{a} & \geq \frac{b-1}{2b} \text{, and we are done.}
\end{align*}

\item
We'll set up a correspondance between these two types of partitions and
prove that it's a bijection.

First, for any partition $\mathcal{P}$ of $n : \mathcal{P} = \sum_{k=1}^n p_k \cdot k$ we can define the doubling function:

\begin{align*}
f(\mathcal{P}) & = \sum_{k=1}^n \left( \sum_{i=1}^{\lceil \log_2(p_k) \rceil} 2^i d_i \right) k \\
& = \sum_{k=1}^n \sum_{i=1}^{\lceil \log_2(p_k) \rceil} d_i (2^i k)
\end{align*}

Here $d_{k_i}$ is the $i^{\text{th}}$ digit in the binary representation of $p_k$. Clearly the inner sum in the first line
must be equal to $p_n$, and so this is another partition of $n$ using addition of $(2^i k)$ rather than $k$.
Let the metric $m(\mathcal{P})$ be the smallest $k$ for which $p_k > 1$
(i.e. the smallest repeated number in the partition). Then $m(f(\mathcal{P})) \geq m(\mathcal{P})$, with equality if
and only if $f(\mathcal{P}) = \mathcal{P}$. As there are only finitely many partitions of $n$ for any given $n$, it is
clear that repeatedly applying $f$ will eventually lead to the ``maximal'' (by the standards of $m$) partition
$\mathcal{P}'$. The process of repeatedly applying $f$ until reaching a fixed point will be denoted by the
function $F(\mathcal{P}) = P'$.

The only fixed points of the binary representation function $p \to \sum 2^i d_i$ are 0 and 1. Therefore if $\mathcal{P}$
is a fixed point of $f$, it must be the case that $p_k \in \{0, 1\} \forall k$; that is, $\mathcal{P}$ is a distinct partition.
This implies that $F(\mathcal{P})$ is a distinct partition.

Using a different observation about positive integers, we can write any $k = 2^{a_k}o_k$, where $o_k$ is an
odd positive integer and $a_k$ is a nonnegative integer. Now we define the halving function $G$:

\begin{align*}
G(\mathcal{P}) & = \sum_{k=1}^n p_k (2^{a_k}o_k) \\
& = \sum_{k=1}^n (2^{a_k}p_k)o_k
\end{align*}

Again, the parenthetical quantity in the first line is equal to $k$, so this is another partition of $n$. As the sum
of only odd numbers, $G(\mathcal{P})$ is an odd partition. When $F$ is restricted to acting only on odd partitions
and $G$ is restricted to only distinct partitions, we claim that $G = F^{-1}$. This is actually easiest to do if you start
with even more atomic $f$ and $g$ doubling and halving functions with $g = f^{-1}$; trying to prove this with
sequences of functions is harder. That said, I've already spent quite a while on this proof so I'm going to leave that
as an exercise for later.

Once the bijection is established, we are done.

\end{enumerate}

\chapter{Generating Functions}

\begin{enumerate}[label={8.\arabic*}]

\item
(a) The parity of $i$ and $n$ must be the same, so 0 if $i \neq n \pmod{2}$. Otherwise, $i$ steps are taken to the
right, and an equal number $(n-i)/2$ of steps are taken to the right and left in addition. The answer in this case is
${n \choose \frac{n-i}{2}} = {n \choose \frac{n+i}{2}}$.

Generating function interpretation: coefficient of $x^i$ in

$$(x^1 + x^{-1})^n = \frac{(1 + x^2)^n}{x^n} = \sum_{k=0}^n {n \choose k} x^{2k-n}$$
$$2k-n = i \iff k = \frac{n+1}{2}$$
$$\text{The desired coefficient is then } {n \choose \frac{n+i}{2}} = {n \choose \frac{n-i}{2}}$$

(b) Looking for the coefficient of $x^i$ in

\begin{align*}
(x^1 + 1 + x^{-1})^n & = \frac{(1 + x + x^2)^n}{x^n} \\
& = \frac{((1+x)^2 - x)^n}{x^n} \\
& = \sum_{k=0}^n (-1)^k {n \choose k} (1+x)^{2k}x^{-k} \\
& = \sum_{k=0}^n (-1)^k {n \choose k} \sum_{\ell=0}^{2k} {2k \choose \ell}x^{\ell-k}
\end{align*}

Now $\ell-k = i \iff \ell = k + i$, so our coefficient is:

$$c_i = {n \choose i}_2 = \sum_{k=0}^n (-1)^k {n \choose k}{2k \choose k+i}$$

Alternatively we can decompose this many other ways, resulting in many expressions for the trinomial coefficients.

(c) This is easier, since $(x + 2 + x^{-1})^n = (\sqrt{x} + \sqrt{x^{-1}})^{2n}$, and we are seeking the coefficient of
$x^i = \sqrt{x}^{2i}$:

$$\left(\sqrt{x} + \frac{1}{\sqrt{x}}\right)^{2n} = \sum_{k=0}^{2n} {2n \choose k}x^{k/2 - (2n-k)/2} = \sum_{k=0}^{2n} {2n \choose k}x^{k - n}$$

Now $k - n = i \iff k = n+i$, so our coefficient is ${2n \choose n+i}$.

\item
The LHS is obvious and comes from the definition of the generating function for $(1+x)^{m+n}$. For the RHS,
apply Proposition 8.1 to get

\begin{align*}
[x^k]\left( (1+x)^m (1+x)^n \right) & = \sum_{i=0}^k \left( [x^i](1+x)^m \right) \left( [x^{k-i}](1+x)^n \right) \\
& = \sum_{i=0}^k {m \choose i}{n \choose k-i}
\end{align*}

\item
\begin{align*}
(x+x^2 + \ldots + x^6)^6 & = x^6(1 + x^3)^6(1 + x + x^2)^6 \\
[x^{21}](x+x^2+\ldots + x^6)^6 & = [x^{15}](1+x^3)^6(1 + x + x^2)^6 \\
& = \sum_{k=1}^5 {6 \choose k}{6 \choose 9-3k}_2 \\
& = \sum_{k=1}^5 {6 \choose k}\sum_{i=0}^6 (-1)^i {6 \choose i}{2i \choose 9-3k+i} \\
& = {6 \choose 1}\cdot 1 + {6 \choose 2}\cdot 50 + {6 \choose 3}\cdot141 \\
& \;\;\;\;\;\;\;\;\;\; + {6 \choose 4}\cdot 50 + {6 \choose 5}\cdot 1 \\
& = 4332
\end{align*}

\item
In each case, every run of tosses is equally likely and Claudia \& Andrea toss the same number of coins as
Adrian \& Zachary, so we just need to show there are the same number of ways for these events to occur.

(i)
\begin{align*}
& \sum_{n=0}^{2004} ([x^n](1+x)^{2004})([x^{n+1}](1+x)^{2004}) \\
= & \sum_{n=0}^{2004} ([x^n](1+x)^{2004})([x^{2004-n-1}](1+x)^{2004}) \\
= & [x^{2003}](1+x)^{4008} \\
= & {4008 \choose 2003} = {4008 \choose 2005}
\end{align*}

(ii) 
\begin{align*}
& \sum_{n=0}^{2003} ([x^n](1+x)^{2003})([x^n](1+x)^{2005}) \\
= & \sum_{n=0}^{2003} ([x^n](1+x)^{2003})([x^{2005-n}](1+x)^{2005}) \\
= & \sum_{n=0}^{2003} {2003 \choose n}{2005 \choose 2005-n} \\
= & {4008 \choose 2005} = {4008 \choose 2003} \text{ (Vandermonde Identity)}
\end{align*}

Both probabilities are given by the binomial $\frac{{4008 \choose 2003}}{2^{4008}}$.

N.B. We also could have treated (ii) like (i), but I wanted to demonstrate both ways of calculation.

\item
\begin{align*}
& {n \choose 0}^2 - {n \choose 1}^2 + \ldots + (-1)^n {n \choose n}^2 \\
= & \sum_{k=0}^n (-1)^k {n \choose k}^2 \\
= & \sum_{k=0}^n (-1)^k {n \choose k}{n \choose n-k} \\
= & \sum_{k=0}^n a_k b_{n-k} \text{, for }a_k = [x^k](1-x)^n, b_k = [x^k](1+x)^n \\
= & [x^n](1-x)^n(1+x)^n \\
= & [x^n](1-x^2)^n
\end{align*}

Which is 0 if $n$ is odd, and if $n$ is even we can write $n = 2m$ to express this quantity as $(-1)^m {2m \choose m}$.

\item
Consider the sum of a subset of $T$ as a single term with coefficient 1 in the generating function

$$ g(x) = \prod_{k=1}^p (1 + x^k) $$

If $p$ divides the sum of the subset, then it divides the $k$ value of the term, or in other words, $k \equiv 0 \pmod{p}$.
We want to add up the coefficients of all such $k$ (including the constant term / empty set!), and we can do so in a
similar manner to Examples 8.11 and 8.12 (I'm assuming, haven't read that one yet):

\begin{align*}
\sum_{j \equiv 0 \pmod{p}} c_j & = \frac{1}{p}\sum_{j=1}^p g(e^{\frac{2\pi i j}{p}}) \\
& = \frac{1}{p}\sum_{j=1}^p\prod_{k=1}^p (1+e^{\frac{2\pi i jk}{p}}) \\
& = \frac{1}{p}\left[\left(\sum_{j=1}^{p-1}\prod_{k=1}^p (1+e^{\frac{2\pi i jk}{p}})\right) + 2^p\right] \\
& = \frac{1}{p}\left[\left(\sum_{j=1}^{p-1} 2\right) + 2^p\right] \\
& = \frac{2^p + 2p - 2}{p}
\end{align*}

\item
(a) Let's try to factor the generating function for the result, similar to Example 8.7:

\begin{align*}
g(x) & = x^2 + x^3 + x^4 + x^5 + x^6 + x^7 + x^8 + x^9 + x^{10} + x^{11} + x^{12} + x^{13} \\
& = x^2(1 + x + x^2 + x^3 + x^4 + x^5)(1 + x^6) \\
& = x^2(1+ x + x^2)(1 + x^3)(1 + x^6) \\
& = x^2(1 + x + x^2 + x^6 + x^7 + x^8)(1 + x^3) \\
& = x^2(1+x + x^2)(1 + x)(1 - x + x^2)(1 + x^2)(1 - x^2 + x^4)
\end{align*}

The second and fourth lines can be rewritten into the product of two generating functions, one for each die roll
of a 6-sided die:

\begin{align*}
& x^2(1 + x + x^2 + x^3 + x^4 + x^5)(1 + x^6) \\
= \;\; & \frac{1}{3}(x^1 + x^2 + x^3 + x^4 + x^5 + x^6)(x^1 + x^1 + x^1 + x^7 + x^7 + x^7)\;\;\;\text{(\#1)} \\
= \;\; & \frac{1}{3}(x^1 + x^2 + x^3 + x^7 + x^8 + x^9)(x^1 + x^1 + x^1 + x^4 + x^4 + x^4)\;\;\;\text{(\#2)}
\end{align*}

Generating function \#1 corresponds to having one normal 6-sided die, and another where 3 of the faces have a 1
and the other 3 faces have a 7. Function \#2 corresponds to a die containing the numbers \{1, 2, 3, 7, 8, 9\} and
another where 1 is written on 3 sides and 4 is written on the other 3 sides.

If we are allowed atypical die, we can also rewrite this generating function to use a 6-sided die with numbers \{1, 1, 2, 2, 3, 3\}
as well as a tetrahedral die with numbers \{1, 4, 7, 10\}.

(b)
\begin{align*}
\sum_{k=2}^{12} \frac{1}{11}x^k = & \left(\sum_{k=1}^6 c_k x^k\right)^2 \\
= & \sum_{k=2}^{12}\left(\sum_{n=1}^6 c_kc_{k-n}\right)x^k
\end{align*}

And this is enough to see it is impossible. For the $k=2, 12$ case we must have $c_1 = c_6 = \frac{1}{\sqrt{11}}$, and by
symmetry we have $c_k = c_{n-k}$ in general. Then the $k=7$ case gives a contradiction:

$$\frac{1}{11} = \sum_{n=1}^3 c_n^2 = \frac{1}{11} + c_2^2 + c_3^2 > \frac{1}{11}$$

Therefore the identical magical dice can't sum to produce a uniform distribution on \{2, 3, \ldots, 12\}.

N.B. The magical dice don't even have to be identical. Dividing both sides of the original equation by $x^2$ gives:

$$\sum_{k=0}^{10} \frac{1}{11}x^k = \sum_{k=0}^{10} \left(\sum_{n=0}^5 a_nb_{k-n}\right)x^k = g_1(x)g_2(x)$$

Where the generating function coefficients of $x^k$ correspond to the number $k+1$ on the die face. By the logic above,
we must have $a_5b_5 = \frac{1}{11}$ for the case where the die sum to 12. However, the polynomial on the left has no
real roots (roots are $\{e^{\frac{2\pi i k}{11}}\}_{k=1}^{10}$, the primitive $11^{\text{th}}$ roots of unity) and so
$g_1$ and $g_2$ may not have real roots either. Therefore they must have even degree, meaning $a_5 = b_5 = 0$,
a contradiction. There can be no two magical dice that sum to produce a uniform distribution of these 11 numbers.

\item
The appearance of the generalized pentagonal numbers hearkens back to Example 4.14 and the result on increasing
partitions of $n$. For now, we will ignore the case that $n$ is a generalized pentagonal number.

In that example, if we call $d_{\mathcal{P}}$ the number of consecutive largest elements of $\mathcal{P}$
(i.e. if $a = (1, 3, 7, 8, 9), \; d_a = 3$ for (7, 8, 9)), this example established
a bijection between $A_1$, the subset in which the smallest element of $\mathcal{P}$ was at most $d_{\mathcal{P}}$,
and $A_1'$, the complement of this subset. In this solution the smallest element is distributed among the largest,
reducing the length of the partition by 1.

Taking the conjugate of these Ferrers (Young) diagrams, this turns increasing partitions into distinct partitions and gives a
one-to-one correspondence between distinct partitions of even length and those of odd length. If we call the action of the
previous bijection on the conjugate diagrams $b$, then for any partition $\mathcal{P}$ we know that $\mathcal{P}$
and $b(\mathcal{P})$ have lengths that differ by 1, and therefore have different even/odd parity.
This implies there are an equal number of even-length distinct
partitions and odd-length distinct partitions, and $[x^n]f(x) = 0, n \neq \frac{3k^2 \pm k}{2}$.

Taking on the case where the element in $A_1$ has no image, we have one element unaccounted for,
corresponding to $n = \sum_{m=k}^{2k-1} m = \frac{3k^2 - k}{2}, k \in \mathbb{N}$.
This has length $k$, and therefore contributes to our generating function with a coefficient of $(-1)^k$.

In the case where an element in $A_1'$ has no preimage, this corresponds to
$n = \sum_{m=k+1}^{2k} m = \frac{3k^2 + k}{2}, k \in \mathbb{N}$. Similarly, this has length $k$ and therefore
contributes $(-1)^k$ to the coefficient of $x^n$ in our generating function.

The proof is done, but I think it's worth mentioning that $\frac{3(-k)^2 - (-k)}{2} = \frac{3k^2 + k}{2}$, that is that
we can consider all integer values (including negatives) of $k$ instead of using the $\pm$ sign.

\item
Using the generating function of the Legendre polynomials

$$\sum_{n \geq 0} P_n(y)t^n = \frac{1}{\sqrt{1-2yt+t^2}}$$

and the known fact

$$P_n(y) = \left[ \frac{y-1}{2} \right]^n \sum_{k=0}^n {n \choose k}^2 \left[ \frac{y+1}{y-1} \right]^k$$

We can substitute $y = -\frac{x^2 + 1}{2x}$ to get

$$P_n\left( -\frac{x^2 + 1}{2x} \right) 
= (-1)^n\frac{(1+x)^2n}{4^nx^n}\sum_{k=0}^n {n \choose k}^2 \left[ \frac{(1-x)^2}{(1+x)^2} \right]^k$$

It follows that we can represent our function as

\begin{align*}
& \sum_{k=0}^n {n \choose k}^2 (1+x)^{2n-2k}(1-x)^{2k} \\
= & (-1)^n4^nx^nP_n\left(-\frac{x^2+1}{2x}\right) \\
= & (-1)^n4^nx^n[t^n]\frac{1}{\sqrt{1 + ((x^2 + 1)/x)t + t^2}} \\
= & [t^n]\frac{1}{\sqrt{1 - 4(x^2 + 1)t + 16x^2t^2}} \\
= & [t^n]\frac{1}{\sqrt{1-4t}}\frac{1}{\sqrt{1-4x^2t}}
\end{align*}

Observing that we must have $\ell = 2r$ for the coefficient to not be identically zero, we continue with

\begin{align*}
& [x^{2r}][t^n]\frac{1}{\sqrt{1-4t}}\frac{1}{\sqrt{1-4x^2t}} \\
= & [x^r][t^n]\frac{1}{\sqrt{1-4t}}\frac{1}{\sqrt{1-4xt}} \\
= & [x^r]\sum_{k=0}^n {2k \choose k}{2n-2k \choose n-k}x^k \\
= & {2r \choose r}{2n-2r \choose n-r}
\end{align*}

Failed first attempt below:

\begin{align*}
& [x^{2r}][y^n]\left((y+1+x^2)^2 - 4x^2 \right)^n \\
= & [x^{2r}]\sum_{k=0}^n {2n-2k \choose k; n-2k} (-4)^{k}x^{2k}(1+x^2)^{n-2k} \;\text{(Multinomial Expansion)} \\
= & [x^{r}]\sum_{k=0}^n (-4)^k {n \choose k}{2n-2k \choose n}x^k(1+x)^{n-2k} \;\text{($x^2 \to x$ substitution)} \\
= & \sum_{k=0}^r (-4)^k{n \choose k}{2n-2k \choose n}{n-2k \choose r-k} \text{, unclear where to go from here.}
\end{align*}

\item
Looking at Example 8.9 for reference, we see that we can obtain all nonnegative integers with the given property
using the generating function $f(x) = \prod_{k \geq 0}(1+x^{4^k})$, as it has the property:

\begin{align*}
f(x)f(x^2) & = \prod_{k \geq 0}(1+x^{4^k})(1 + x^{2\cdot4^k}) \\
& = \prod_{k \geq 0}\frac{1-x^{4^{k+1}}}{1-x^{4^k}}\text{, which telescopes to:} \\
& = \lim_{n \to \infty} \frac{1 - x^n}{1-x} \\
& = x^0 + x^1 + x^2 + \ldots \text{, as desired.}
\end{align*}

The key observation to adapt this to all integers is $1 - x^{-k} = \frac{x^k - 1}{x^k}$. We cannot make every
exponent negative (this would effectively generate all nonpositive integers instead), but we can make every
\emph{other} exponent negative with $g(x) = \prod_{k \geq 0}(1+x^{(-4)^k})$:

\begin{align*}
g(x)g(x^2) & = \prod_{k \geq 0}(1+x^{(-4)^k})(1 + x^{2\cdot(-4)^k}) \\
& = \prod_{k \geq 0}\frac{1-x^{4^{k+1}}}{1-x^{4^k}}\text{, which telescopes to:} \\
& = \lim_{n \to \infty} \frac{1 - x^{16^n}}{x^{\frac{4(16^n - 1)}{5}}(1-x)} \\
& = \lim_{n \to \infty} \frac{1}{x^n (1-x)} \\
& = \ldots + x^{-2} + x^{-1} + x^0 + x^1 + x^2 + \ldots \text{, as desired.}
\end{align*}

We have proved that such an $X$ exists, and from our generating function it is all integers which have a
representation in base -4 of only ones and zeros.

There must be a combinatorial argument for this as well, and it would be nice to include because the 
generating function algebra was neither clean nor obvious.

\item
Let us number these vertices $\{0, \ldots, n-1\}$ and consider the coloring of the vertices as the splitting
of a generating function:

\begin{align*}
f(x) & = x^0 + x^1 + \ldots + x^{n-1} = \sum_{k=0}^{n-1}x^k \\
& = f_1(x) + f_2(x) + \ldots + f_\ell(x) \text{, where} \\
f_1(x) & = \left(x^{s_1} + x^{s_1 + d_1} + x{s_1 + 2d_1} + \ldots + x^{s_1 + \left(\frac{n}{d_1} - 1\right)}\right) \\
& = x^{s_1}\sum_{i=0}^{\frac{n}{d_1}-1} x^{d_1 i} \\
& \text{with } s_1 < d_1 ; \; d_1 | n ; \; \text{and all }f_i\text{ are defined similarly.}
\end{align*}

Let us write $n_i = n/d_i$ to keep track of the number of vertices in each regular polygon.
In order to achieve a contradiction, assume that the $n_i$ are all distinct, and order
the functions so that $n_1 < n_2 < \ldots < m_\ell$. Then we can consider $f(x^{n_1})$, which satisfies:

\begin{align*}
f_i(x^{m})|_{x = e^{\frac{2\pi i}{n}}} = 0 \text{ if } m \not| n_i \text{; else }n_ie^{\frac{2\pi i s_i m}{n}} \\
\end{align*}

Therefore $f(x^{n_1}) = 0$, because $n_1 < n$, and $f_i(x^{n_1}) = 0$ for all $i > 1$ for the same reason.
However, this gives:

\begin{align*}
0 & = f(x^{n_1}) = f_1(x^{n_1}) + \sum_{i=2}^\ell f_i(x^{n_1}) \\
& = n_1e^{\frac{2\pi i s_1 n_1}{n}} + 0 \text{, a contradiction.}
\end{align*}

\item
It must be the case that the author means

$$B_{n+1} = (A_n \cup B_n) \setminus (A_n \cap B_n)$$

Other interpretations of this recursion make no sense. If we observe the behavior of the recurrence,
we can write a relation for the generating function of $B_n$ with:

$$f_1 (x) = f_2(x) = 1; \; f_{n+2}(x) = f_{n+1}(x) + xf_n(x)$$

Moreover, we must interpret the coefficients $\pmod{2}$. By setting \\
$g(x, t) = \sum_{n=1}^{\infty}f_n(x)t^n$, we can write

\begin{align*}
& g(x, t) - t - t^2 = (t + xt^2)g(x, t) - t^2 \\
& g(x, t)(1-t-xt^2) = t \\
& g(x, t) = \frac{t}{1-t-xt^2} \\
& \;\;\;\;\;\;\;\;\;\; = \sum_{n=0}^{\infty} \frac{(1+\sqrt{1+4x})^n - (1-\sqrt{1+4x})^n}{2^n \sqrt{1-4x}}t^n \\
& \;\;\;\;\;\;\;\;\;\; = \sum_{n=0}^{\infty} \left(2^{1-n} \sum_{k=0}^{\lfloor \frac{n-1}{2}\rfloor} {n \choose 2k+1}(1+4x)^k \right)t^n
\end{align*}

If we could substitute $n = 2^m$ and find that $2 | [x^k][t^{2^m}]g(x, t)$ for all $k$, then we would have proved that
$B_{2^m} = \{0\}$ for all $m$ (after a short induction that $0 \in B_n \;\; \forall n$).

\end{enumerate}

\chapter{Review Exercises}

\begin{enumerate}[label={9.\arabic*}]

\item
You can see the outside of one half, that is, $11^3 - 10^3 = 331$ unit cubes at most.

Alternatively, you can use PIE on the 3 visible faces. There are $11^2$ cubes per face, 11 cubes per edge (common
to 2 faces), and one corner (common to all 3 faces), so the total is $3\cdot11^2-3\cdot11+1 = 331$ visible unit cubes.

\item
Any two adjacent evens or odds will have a sum divisible by 2, and the sums divisible by 3 but not by 2 are
$1+2=3$ and $3+6 = 4+5=9$, so 1 and 2 / 4 and 5 / 3 and 6 must not be adjacent to each other. This is the
\emph{probl\`eme des m\'enages} from Chapter 6, with evens/odds instead of men and women,
and the pairs \{(1, 2), (3, 6), (4, 5)\} instead of couples. From the solution there, we have:

\begin{align*}
M_3 & = 2\cdot3!(3! - \frac{6}{5}{5 \choose 1}2! + \frac{6}{4}{4 \choose 2}1! - \frac{6}{3}{3 \choose 3}0!) \\
& = 12(6-12+9-2) = 12
\end{align*}

Alternatively, we can count cases. The symmetry is such that we can count arrangements starting with 1 and then
multiply by 6.

Case 1: Start with (1, 4). Then we must continue with (1, 4, 3, 2, 5, 6).

Case 2: Start with (1, 6). This is the previous case in reverse.

Therefore there are $6 \cdot 2 = 12$ total arrangements.

\item
\begin{align*}
n & = \lfloor \frac{2003}{5} \rfloor + \lfloor \frac{2003}{5^2} \rfloor + \lfloor \frac{2003}{5^3} \rfloor + \lfloor \frac{2003}{5^4} \rfloor \\
& = 400 + 80 + 16 + 3 = 499
\end{align*}

\item
Direct counting or case counting with bijection both work here.

DC: the two digit number $\overline{ad}$ ``contains'' $\overline{ad}$ many numbers (00 to $\overline{a(d-1)}$), so we have

$$ \sum_{n=10}^99 n = 50\cdot99 - 5\cdot9 = 4950 - 45 = 4905 \text{ containing numbers.}$$

CCwB: Among the $9000$ 4-digit numbers, there are 3 cases.

Case 1: $\overline{bc} < 10$. In this case, $\overline{ad}$ ``contains'' $\overline{bc}$ no matter what. There are 900 of these.

Case 2: $\overline{ad} \neq \overline{bc}$. In this case, we can establish a bijection by switching
$\overline{ad}$ and $\overline{bc}$. Half of these numbers will be ``contained''.

Case 3: $\overline{ad} = \overline{bc}$. There are 90 possible values for $\overline{ad}$, so there are 90 in this case. This gives

$$900 + \frac{1}{2}(9000 - 900 - 90) = 4905 \text{ containing numbers, as before.}$$

\item
There are 3 ways you can start the word, and then at each new letter there are 2 ways to continue the word.
Therefore the answer is $3\cdot 2^6 = 192$ seven-letter good words.

\item
For an $i \times j$ sub-chessboard, there are $m-i+1$ places for the rows to fit and $n-j+1$ places for the columns
to fit. Then, assuming we don't care about black/white parity, square size, or anything other than rectangularity,
we have:

\begin{align*}
& \sum_{i=1}^m \sum_{j=1}^n (m-i+1)(n-j+1) = \frac{m(m-1)n(n-1)}{4}
\end{align*}

as the number of sub-chessboards, including the original.

\item
$${10 \choose 5} = 252 \text{ times}$$

\item
$q + r$ is divisible by 11 if and only if $n$ is. $10008 = 11 \cdot 912$ is the smallest 5-digit multiple of 11, and
$99990 = 11 \cdot 9090$ is the largest, so there are $9090 - 912 + 1 = 8179$ such values of $n$.

\item
To get to the bottom, you must go through either $A_{10}$ or $A_{11}$. The graph is symmetrical, so let's
count the ways to get to $A_{10}$. You can come through $A_5$ in ${2 \choose 0} = 1$ way, through $A_6$ in
${3 \choose 1} = 3$ ways, and through $A_7$ in ${4 \choose 2} = 6$ ways, for a total of 10 ways. This makes
$10^2 = 100$ ways to get to $A_{20}$ through $A_{10}$, and double this for coming through $A_{11}$ to get
a total of 200 ways for Mr. Fat to travel from $A_1$ to $A_{20}$ moving downward with each step.

\item
There are $9{4 \choose 2} + {2 \choose 2} = 56$ possible pair draws remaining out of ${38 \choose 2} = 703$ possible
draws of 2 cards. Therefore the probability is $\frac{56}{703}$.

\item
$$F_n \text{ ways}$$

\item
PIE helps here.

\item
The answer will either be 5 or 0 (if it isn't possible to have $n!$ end in 2003 zeros). In any case, we have $n$ is nearly
$2003 \cdot 5 = 10015 \in (5^5, 5^6)$. Referring back to Problem 9.3, we have

$$2003 = \lfloor \frac{n}{5} \rfloor + \lfloor \frac{n}{5^2} \rfloor + \lfloor \frac{n}{5^3} \rfloor + \lfloor \frac{n}{5^4} \rfloor + \lfloor \frac{n}{5^5} \rfloor = f(n)$$
$$f(10000) = 2000 + 400 + 80 + 16 + 3 = 2499$$
$$f(7600) = 1520 + 304 + 60 + 12 + 2 = 1898$$
$$f(8100) = 1620 + 324 + 64 + 12 + 2 = 2022$$
$$f(8025) = 1605 + 321 + 64 + 12 + 2 = 2004$$
$$f(8020) = 1604 + 320 + 64 + 12 + 2 = 2002$$

And we see that there are no values of $n$ for which $f(n) = 2003$, and therefore there are 0 solutions.

\item
There is almost certainly a cleverer way of doing this (it's reminiscent of the late Chapter 3 examples), but
here's a brute force method:

\begin{align*}
\frac{{n \choose k+1}}{{n \choose k}} & = \frac{n-k}{k+1} = \frac{4}{3} \iff 3n-7k=4 \\
\frac{{n \choose k+2}}{{n \choose k+1}} & = \frac{n-k-1}{k+2} = \frac{5}{4} \iff 4n-9k=14 \\
\begin{bmatrix} n \\ k \end{bmatrix} & = \begin{bmatrix} 3 & -7 \\ 4 & -9 \end{bmatrix}^{-1} \begin{bmatrix} 4 \\ 14 \end{bmatrix} \\
& =  \begin{bmatrix} -9 & 7 \\ -4 & 3 \end{bmatrix} \begin{bmatrix} 4 \\ 14 \end{bmatrix} = \begin{bmatrix} 62 \\ 26 \end{bmatrix}
\end{align*}

We get $n=62$ from solving this linear system.

\item
$$2\sum_{k=0}^{5} {6 \choose k} + 1 = 2(2^6 - 1) + 1 = 2^7 - 1 = 127 \text{ paths}$$

\item
\begin{align*}
& \;\;\;\;\; \sum_{i=0}^n {n \choose i} \sum_{j=i}^n {n \choose j} \\
& = \sum_{0 \leq i \leq j \leq n} {n \choose i}{n \choose j} \\
& = \frac{1}{2}\left(\sum_{0 \leq i, j \leq n} {n \choose i}{n \choose j} + \sum_{k=0}^n {n \choose k}^2 \right) \\
& = \frac{1}{2}\left((2^n)^2 + {2n \choose n}\right) \\
& = \frac{4^n + {2n \choose n}}{2} \\
\end{align*}

Or alternatively:

\begin{align*}
& \;\;\;\;\; \sum_{i=0}^n {n \choose i} \sum_{j=i}^n {n \choose j} \\
& = \sum_{i=0}^n \sum_{j=0}^i {n \choose i} {n \choose j} \\
& = \sum_{i=0}^n \sum_{j=0}^i {n \choose i} {n \choose i-j} \\
& = \sum_{i=0}^n {2n \choose i} \\
& = \frac{1}{2}\left( \left[\sum_{i=0}^{2n} {2n \choose i}\right] + {2n \choose n} \right) \\
& = \frac{4^n + {2n \choose n}}{2}
\end{align*}

\item
The basic answer is that if $n$ is even there are (many!) good arrays, and if $n$ is odd there are not. My solution
is more of an invariant-style than combinatorial, and I should rethink this to come up with a combinatorial argument.

The invariant in this case is the congruency class of the sum of products $\pmod{4}$. Consider changing a single
element (it doesn't matter whether you change from 1 to -1 or the other way around, in either case the effect
is flipping the sign), which changes the sign of both the row and the column that element contains. This has the
effect of changing the row sum by 2 and the column sum by 2, meaning that the total change in the sum is in
$\{-4, 0, 4\} \equiv 0 \pmod{4}$.

By this invariant, we can show that all sums are equivalent to $2n \pmod{4}$, because the sum of the
products of a matrix where all entries are 1 is $2n$. If $n$ is even, then $2n \equiv 0 \pmod{4}$ and it is possible
for a good array to exist (e.g. changing the first $\frac{n}{2}$ elements along the main diagonal to be negative).
If $n$ is odd, then $2n \equiv 2 \not\equiv 0 \pmod{4}$, and a good array cannot exist.

\item
If $k=1$, we can define a bijection between the partitions making up $p(n, m, 1)$ and $p(n-1, m-1)$ by removing
a single 1 from the partition.

If $k > 1$, we can define a similar bijection by removing 1 from every element to get $p(n, m, k) = p(n-m, m, k-1)$.

\item
There are ${8 \choose 4} = 70$ ways to choose the 4 people to move around, and $!4 = 9$ ways to switch all of their
seats. This is assuming that, of the 4 people who move, all are required to end up in a different seat than they started.
In this case, the answer is $70 \cdot 9 = 630$ ways.

\item
There are $4^n$ n-digit numbers with this property, so we get

$$\sum_{n=1}^6 4^n = \frac{4^7 - 4}{3} = 5460 \text{ such integers.}$$

\item
I'll case count here as well, where the cases involve the square being of side length 1, the square not containing any
points in the second quadrant (including x-axis), and the 4 cases of exactly one vertex in the second quadrant.

Case 1: Side length is 1. We have a width and height of 6, so PIE gives $6+6-1 = 11$ squares in this case.

Case 2: No points in Q2, side length > 1. There are 2 squares in this case: \{(0, 1), (1, 2), (2, 1), (1, 0)\} and
\{(0, 0), (1, 1), (2, 0), (1, -1)\}.

Case 3: One vertex in Q2. There are no squares with exactly two or 3 vertices in Q2 with side length greater than 1,
which can be seen by checking each case. In all of these cases except for a single square \{(-1, 1), (0, 2), (1, 1), (0, 0)\},
one of the vertices must also be in Q4 (including y-axis), so there are up to 4 other squares for each of the following
subcases.

Case 3a: (-1, 0) makes a square with (0, -1), (1, -1), and (1, -2), so there are 3 squares here.

Case 3b: (-2, 0) makes squares with (0, -1), (1, -1), and (1, -2), so there are 3 squares here.

Case 3c: (-1, 1) makes squares with (0, 2), (0, -1), and (1, -1), 3 squares here.

Case 3d: (-2, 1) makes squares with (0, -1), (0, -2), and (1, -2), 3 squares here.

The total is then $11 + 2 + 3 + 3 + 3 + 3 = 25$ squares in all.

\item
There are ${9 \choose 2} = 36$ ways of choosing 2 vertices of the nonagon and 2 ways to orient the third vertex
of the triangle, for 72 total ways. However, some of these triangles are identical, so we need to use PIE to account
for this. There are 3 equilateral triangles you can form from the vertices of a nonagon, so our final answer is

$${9 \choose 2} \cdot 2 - {2 \choose 1} \cdot 3 = 66 \text{ distinct triangles}$$

\item

Fat has 3 choices of edge initially, then 2 choices, then 1 choice, for a total of $3! = 6$ ways.

Alternatively, each point that shares an edge with vertex $G$ shares a face with vertex $A$ across the diagonal.
There are 3 points that share an edge with vertex $G$ and there are 2 ways to traverse a face across the diagonal,
for a total of $3 \cdot 2 = 6$ ways. This second argument also works in reverse.

\item
Rewrite $f(x) = \frac{1-x^{18}}{1+x}$, and $x = y-1$ to get

\begin{align*}
f(y) & = \frac{1-(y-1)^{18}}{y} \\
& = \frac{1 - \sum_{k=0}^{18} {18 \choose k}(-1)^ky^k}{y} \\
& = \sum_{k=0}^{17} {18 \choose k}(-1)^{k+1}y^{k-1} \\
a_2 & = [y^2]f(y) = {18 \choose 3} = 816
\end{align*}

\item
A distinct partition either contains 1 one time, or not at all. If you subtract 1 from every number in a partition of $n$
of length $m$, it becomes a partition of $n-m$ of length $m$ (if the partition does not contain 1) or $m-1$ (if it does).
This is the bijection that gives the identity

$$d(n, m) = d(n-m, m) + d(n-m, m-1)$$

\item
There are 48 vertices total ($12 \cdot 4 = 8 \cdot 6 = 6 \cdot 8 = 48$), and from Euler's Formula we can determine the
number of edges:

$$V - E + F = 2 \iff E = 48 + 26 - 2 = 72$$

Then there are $n(n-3)/2$ face diagonals per each regular $n$-gon face. Subtracting out these from the total gives

\begin{align*}
D & = {48 \choose 2} - 12\cdot\frac{4\cdot1}{2} - 8\cdot\frac{6\cdot3}{2} - 6\cdot\frac{8\cdot5}{2} - 72 \\
& = 24\cdot47 - 24 - 24\cdot3 - 24\cdot5 - 24\cdot3 \\
& = 24\cdot35 = 840 \text{ interior diagonals in total.}
\end{align*}

\item
If we call $s_k$ the number of new servers in building $k$, then every pair of servers will be connected except for
those in the same building, so the number we're looking for is

\begin{align*}
C & = {25 \choose 2} - \sum_{k=1}^{13} {s_k \choose 2} \\
& \geq {25 \choose 2} - {13 \choose 2} - 12{1 \choose 2} \text{ (Jensen's Inequality)} \\
& = 300 - 78 - 12\cdot0 = 222 \text{ connections at minimum.}
\end{align*}

Alternative, again using Jensen's Inequality:
\begin{align*}
\sigma_2 & = \frac{\sigma_1^2 - p_2}{2} = \frac{625 - p_2}{2} \geq \frac{625 - 1 \cdot 13^2 - 12 \cdot 1^2}{2} = \frac{444}{2} = 222
\end{align*}

\item
Starting out, there are ${6 \choose 2}$ ways of choosing the empty boxes. After that, there are ${8 \brace 4}$ ways
to partition 8 objects into 4 nonempty subsets, where this is the Stirling number of the second kind:

$${n \brace k} = \frac{1}{k!}\sum_{i=0}^k (-1)^i {k \choose i}(k-i)^n$$

Finally, we assign a distinguishable box to each of these partitions in $4!$ many ways, for a final answer of

$${6 \choose 2}{8 \brace 4}4! = 15\cdot1701\cdot24 = 612360 \text{ ways in all.}$$

\item
There are ${4 \choose 2} = 6$ pairs of numbers, and each number must appear at least twice in the sequence, as
each number has 3 pairs. This means that a sequence of length 8 is hypothetically the shortest, and testing:

$$1, 2, 3, 4, 3, 1, 4, 2 \text{ is such a sequence.}$$

\item
Label the points $A, B, C, D, E, F$ such that $\{A, D\}, \{B, E\}, \{C, F\}$ are opposing pairs. Then, because the complements
of equilateral triangles formed by these points are themselves equilateral triangles or else they are collinear, this problem
is identical to finding four times the number of incongruent scalene triangles (twice because reflection is not allowed
for equivalent coloring schemes, and twice for switching the coloring scheme). There are two such triangles,
$\Delta ABD$ and $\Delta ABE$, but these are really a single solution because the complement of $\Delta ABD$ is
$\Delta CEF \sim \Delta ABE$. Therefore there are 4 distinct coloring schemes.

\item
Let $n_p$ be the greatest integer less than or equal to $n$ that is divisible by $p$; that is, $n_p = p\left\lfloor \frac{n}{p} \right\rfloor$. Then:

$${n \choose p} = \frac{n(n-1)\cdots (n-p+1)}{p\cdot (p-1)!} \equiv \frac{n_p}{p}\frac{\prod_{k \in \mathbb{Z}_p^{\times}} k}{\prod_{k \in \mathbb{Z}_p^{\times}} k} \equiv \left\lfloor \frac{n}{p} \right\rfloor \pmod{p}$$

\item
The digit 0 can't be a part of any ascending integer. If there are $k$ digits in an ascending integer ($k \geq 2$), then
each digit is different. There are ${9 \choose k}$ ways of choosing the digits and only one way of ordering them, so
there are

$$\sum_{k=2}^9 {9 \choose k} = 2^9 - 10 = 502 \text{ ascending integers.}$$

\item
This number is limited by both the size of the possible values of $D_{\pi} \in \mathcal{P}(S)$, which is $|\mathcal{P}(S)| = 2^n$, and $|S_n| = n!$. For $n \geq 4$, the limiting factor will be $2^n$.

The first thing to prove is that it is impossible for $A_i = D_{\pi}$ for any $i, \pi$. Clearly this is true for the identity
permutation $\pi_0$, and then we can use the fact that

$$D_{\pi}(A_1, \ldots, A_n) = D_{\pi_0}(A_{\pi(1)}, \ldots, A_{\pi(n)}) \neq A_{\pi(i)} \text{ for all }i.$$

Then, since $\{A_i\} = \{A_{\pi(i)}\}$, this result follows. Aside from the $A_i$, there is no subset of $S$ which isn't
attainable. In fact, letting $A_i = i$ allows you to construct every subset as a $D_{\pi}$ except for the single-element
subsets, giving the answer as $2^n - n$.

N.B. For the cases $n \in \{1, 2, 3\}$, where $2^n > n!$, we can notice that \\
$2^n - n = n!$ and our stricter limit is not an issue.

\item
Of the prime powers, the primes are distinguishable but the powers are not. Factoring $2310 = 2\cdot3\cdot5\cdot7\cdot11$,
we see we have no powers to worry about and we can instead think about partitioning these 5 distinguishable objects
into (up to) 3 indistinguishable containers. This can be accomplished with Stirling numbers of the second kind or case counting.

Case 1: The most primes in a specific variable is 2. In this case, there are 2, 2, and 1 prime in different boxes, and can be
split up in ${5 \choose 2;2;1} = \frac{5!}{2!2!1!} = 30$ ways, though we have doubly counted because $a$ and $b$ contain
the same number of primes. The total in this case is 15 ways.

Case 2: The most primes in a specific variable is 3. In this case we can have $3, 2, 0$ or $3, 1, 1$. There are ${5 \choose 3} = 10$ in the former and ${5 \choose 3;1;1} = \frac{5!}{3!1!1!} = 20$ possibilities in the latter, though this is doubly counted similar
to Case 1. The total in this case is $10 + 20/2 = 20$ ways.

Case 3: The most primes in a specific variable is 4 or 5. This gives ${5 \choose 4} + {5 \choose 5} = 6$ possibilities in this case.

Adding them up, we have 41 ways in total.

Alternatively, we can sum the Stirling numbers of the second kind ${5 \brace k}$, the number of ways to split 5
distinguishable objects into exactly $k$ containers. $\sum_{k=1}^3 {5 \brace k} = 1 + 15 + 25 = 41$ ways in total, as before.

\item
Because $i^n \in \{1, i, -1, -i\}$, we have 3 possible real-valued results for $i^x + i^y$.

Case 1: $i^x + i^y = -2$. In this case we have $i^x = i^y = -1$, so $x \equiv y \equiv 2 \pmod{4}$. There are 25 such numbers
in the given range, so there are $25^2 = 625$ cases here.

Case 2: $i^x + i^y = 0 \implies i^x = -i^y \implies x - y \equiv 2 \pmod{4}$. For any choice of $x$, there are 25 choices of $y$
to satisfy this equation, so there are $100\cdot25 = 2500$ cases here.

Case 3: $i^x + i^y = 2 \implies x \equiv y \equiv 0 \pmod{4}$. As in Case 1 there are 625 possibilities.

This gives a total of $625 + 2500 + 625 = 3750$ such ordered pairs.

Alternative:

We can also solve this probabilistically, since all events are equally likely. There is a $1/4$ probability that the sum
will be 0, and any other off-axis sum (such as $1 + i$) is twice as likely than an on-axis sum. The probability of being on any
axis (not the origin) is $\frac{1}{3}\frac{3}{4} = \frac{1}{4}$, so the probability of being on the $x$-axis is $1/8$. Our
total number of events is $100^2 = 10000$, so our number of ordered pairs is $10000\cdot 3/8 = 3750$.

\item
First, we factor: $20! = 2^{18}\cdot3^8\cdot5^4\cdot7^2\cdot11\cdot13\cdot17\cdot19$, for 8 distinct primes in the
factorization. Alternatively we could just count $\pi(20) = 8$. Given the fraction is written in lowest terms, there are
no primes common to the numerator and denominator. Put another way, this is requiring the primes to be split between
numerator and denominator. There are $2^8 = 256$ ways to do this, and each one has a unique rational number
corresponding to it because it has to be between 0 and 1. Therefore the answer is 256.

\item
This is a good case for recursion. Let $c_n$ be the number of cuts required for $n$ points, $n \geq 5$, and then imagine
adding the $(n+1)^{th}$ point. It is in the interior of one of the triangles, so there are 3 vertices to connect it to, and we
have

$$c_{n+1} = c_n + 3 \implies c_n = 3(n-5) + c_5$$

And we count 4 cuts for the exterior of the square, 4 cuts for the edges connecting to the interior point, for $c_5 = 8$
and therefore

$c_{2003} = 3(1998) + 8 = 6002$ cuts are required.

\item
If $n$ is the number of players in the tournament, there are ${n \choose 2}$ total points earned. Of these, the ten
players with the least points played each other ${10 \choose 2} = 45$ times, and therefore these players earned
90 points in total, with 45 coming against the better-performing players. There are ${n-10 \choose 2}$ games between
these players, and they account for half of the total points minus 45:

$${n-10 \choose 2} = \frac{1}{2}{n \choose 2} - 45$$
$$\frac{n^2 - 21n + 110}{2} = \frac{n^2 - n}{4} - 45$$
$$n^2 - 41n + 400 = 0$$
$$(n-16)(n-25) = 0$$

If there are 16 players, there are 120 points in total, and so it is impossible that the worst-performing 10 players earn
90 points (i.e. the worst $5/8$ of the players earn $3/4$ of the points). Therefore it must be the case that $n = 25$.

\item
I assume this question is meant to ask about subsets of $\{1, 2, \ldots, 2003\}$ instead of subsets of $T$, because
otherwise this question isn't answerable without knowing $T$. We can accomplish this by case counting or PIE.

Case counting:

Case 1: Our subset is 2 relatively isolated groups of 3 and 2 elements, respectively. There are 2001 ways of choosing
a consecutive group of 3 integers, and in all but 4 cases there are 1996 ways of choosing a consecutive pair of integers
that does not border our group of 3.

The aberrant 4 cases are (1, 2, 3), (2, 3, 4), and their images under the reflection $f(k) = 2004 - k$. These
cases have 1998 and 1997 ways of choosing the pair, respectively, for a total of 6 added ways over the usual case across
all special cases.

Case 2: Our subset is a consecutive group of 5 integers, which can happen in 1999 ways. Therefore the total number across
all cases is

$$2001\cdot1996 + 6 + 1999 = 3996001 \text{ non-isolated five-element subsets.}$$

N.B. This is $1999^2$, which suggests a different solution. Placing a 3- and a 2-element consecutive subset is equivalent
to two partitions of the remaining 1997 elements (removing one for the requirement that the subsets not be adjacent)
into a sum of 3 nonnegative integers (two for the ordering of the subset
size) in $2{1999 \choose 2} = 1999\cdot1998$ ways. Placing a single 5-element consecutive subset is equivalent to a partition of
1998 elements into a sum of 2 nonnegative integers in ${1999 \choose 1} = 1999$ ways. Then
$1999\cdot1998 + 1999 = 1999^2$ as before.

\item
The only way a color scheme is not equivalent with 3 other color schemes is if the two squares are opposite across the
center of the board. There are $\frac{7^2 - 1}{2} = 24$ such arrangements that are only equivalent with 1 other color scheme.

$$\frac{{49 \choose 2} - 24}{4} + 24 = \frac{48^2}{8} + 24 = 312 \text{ inequivalent color schemes in all.}$$

\item
This is equivalent to filling 12 spaces among 3 offices, where 3 spaces will remain empty. If the offices are
distinguishable this is like the number of ways to partition 3 into 3 nonnegative integers: ${5 \choose 2} = 10$ ways.

If the offices are indistinguishable, this is the number of ways to partition 3 into at most 3 groups: $p(3) = 3$.

\item
Consider this recursively: for large enough $n$, if you remove $n$ from the set you have a partition of $\{1, \ldots, n-1\}$
that satisfies the problem conditions. Then you can add $n$ to one of two sets in order to achieve a good partition of
$\{1, \ldots, n-1\}$. Therefore, if $a_n$ is the number of such partitions, we have $a_n = 2a_{n-1}$.

If the sets are distinguishable, we have the base case $a_1 = 3$ giving $a_{2003} = 3\cdot2^{2002}$.

If the sets are indistinguishable, we have the base case $a_2 = 1$ giving $a_{2003} = 2^{2001}$.

\item
Referring back to Problem 9.13, we have the zero-counting function

$$f(n) = \lfloor \frac{n}{5} \rfloor + \lfloor \frac{n}{5^2} \rfloor + \lfloor \frac{n}{5^3} \rfloor + \lfloor \frac{n}{5^4} \rfloor + \lfloor \frac{n}{5^5} \rfloor$$

In general, this increments by 4 at every multiple of 625, else by 3 at every multiple of 125, else by 2 at
every multiple of 25, else by 1 at every multiple of 5, else by 0. It skips the values

$$\{n + 6k : k \in \{0, 1, 2, 3, 4, 5\}, n \in \{-1, 30, 61, 92, 123, 155, 186, 217,\ldots \}\} \cup \{154, \ldots\}$$

\item
In order to tie, all 3 letter places must flip in exactly one of the words. There are $2^3 = 8$ ways for this to happen,
and each happens with equal probability $(\frac{1}{3})^3 (\frac{2}{3})^3 = \frac{8}{729}$, so the probability of
a tie is $\frac{64}{729}$. In the remaining $\frac{665}{729}$ cases, consider the first place where the two strings
differ. The likelihood of both strings being correctly received is 4 times the likelihood of both being flipped, so our
answer is

$$p = \frac{4}{5}\cdot\frac{665}{729} = \frac{532}{729}$$

\item
With these numbers being coprime, there is no danger of overcounting in the case of indistinguishable people.
If they are indistinguishable, we can seat a girl at the head of the table, and consider the 25 boys with
dividers of 7 girls. The number of ways to do this is equivalent to the number of ways to divide $25 - 2\cdot(7+1)
= 9$ into a sum of 8 nonzero integers, or ${16 \choose 7}$. The final count in this case takes into consideration
the 33 ways to rotate the table, for a total of $33\cdot{16 \choose 7} = 377520$.

If they are distinguishable, there are 8! ways to rearrange the girls and 25! ways to rearrange the boys, so we
multiply our previous answer by those to get 236105534067311213572364697600000000.

\item
In general, 2 digits can sum to $n$ in $10 - |n - 9|$ many ways. In the case of the first 2 digits of an integer between
1000 and 9999, this becomes $n$ if $n < 10$, $19-n$ if $n \geq 10$ because the first digit can't be zero. Our answer is
then

$$\sum_{n=1}^9 n(n+1) + \sum_{n=10}^{18} (19-n)^2 = \sum_{n=2}^{10} 2{n \choose 2} + \sum_{n=1}^9 n^2 = 2{11 \choose 3} + \frac{9(10)(19)}{6} = 615$$

\item
Let $A \subseteq X; |A| = a \leq n$. There are ${n \choose a}$ choices of $A$.
Then for any subset $B$ disjoint from $A$, we have $B \in \mathcal{P}(X\setminus A)$,
i.e. there are $2^{n-a}$ choices of disjoint $B$. Counting this way, there are

$$\sum_{a=0}^{n} {n \choose a} 2^{n-a} = 3^n \text{ disjoint pairs of sets in all.}$$

(Another way to see this is that for each element of $X$, it either belongs to $A$, to $B$, or to neither.)

Then, letting the size of the intersection be $k$, the desired sum is

$$\sum_{k=0}^n {n \choose k}3^{n-k}\cdot k = 4^{n-1}\cdot n$$

N.B. Dividing by the $4^n$ pairs of subsets gives an average of $n/4$ elements per pair intersection. By the same logic
as we arrived at the $3^n$ disjoint pairs above, in addition to linearity of expectation, we can say that there is a
$1/4$ chance that any element will belong to both $A$ and $B$, and therefore the sum is $\frac{n}{4}4^n = 4^{n-1}\cdot n$
as before.

\item
$0.\overline{abc} = \frac{\overline{abc}}{999}$, and we can factor $999 = 3^3 \cdot 37$. All $\phi(999) = 3^2(3-1)\cdot36 = 648$
numbers less than 999 that are coprime to it will show up as numerators, and only those multiples of 3 that are less
than 37 (i.e. $\frac{3}{37}, \ldots, \frac{36}{37}$) will show up as numerators. Since $37 > 3^3$, no multiples of 37 will
show up as numerators. This gives a total of $648 + 12 = 660$ different numerators.

\item
Using PIE, we see that the intersection of two sets of these integers appearing consecutively is the case when
all 4 appear together. There are ${4 \choose 3} = 4$ ways of choosing three of these, 3! ways of rearranging them,
$(n-2)$ places to put them, and $(n-3)!$ ways of rearranging the other $n-3$ elements of the set. In the intersection
of these sets, when all four of these appear consecutively, there are $4!$ ways of rearranging these elements,
$(n-3)$ places to put them, and $(n-4)!$ ways of rearranging the other elements. In total we have:

\begin{align*}
& 4\cdot 3! \cdot (n-2)\cdot (n-3)! - 4!\cdot(n-3)\cdot(n-4)! \\
 = \; & 24(n-3)!((n-2) - 1) = 24(n-3)(n-3)!
\end{align*}

\item
There are 14 connections and 14 possibilities for $|a - b|$, so each of these absolute values must be represented.
Each of the numbers is connected to an even number of other numbers, and therefore if we sum all of the
absolute value differences we will get an even number. However, \\
$\sum_{k=1}^{14} k = 105$, so this is not possible.

\item
Initially there are $1024 = 2^{10}$ closed lockers.
After walking down the hall and returning, only numbers equivalent to $2 \pmod{4}$ remain closed. The transformation
$f(n) = \frac{n-2}{4} + 1$ then turns this into numbers 1 through $256 = 2^8$. Another 3 trips down the hall and another
3 applications of $f$ later and we have just $\{1, 2, 3, 4\}$, leaving 2 as the final number. The answer is then

\begin{align*}
f^{-1}(f^{-1}(f^{-1}(f^{-1}(2)))) & = f^{-1}(f^{-1}(f^{-1}(6))) \\
& = f^{-1}(f^{-1}(22)) \\
& = f^{-1}(86) = 342
\end{align*}

\item
As the length and width are both even, two of the four corners of this chessboard contain white squares. Orient
the board so that they are in the top left and bottom right. Considering the board like a matrix, fill the main
diagonal with 2002 ones. Next, fill the three white squares $M_{2, 2002}, M_{1, 2003}, M_{2, 2004}$ with ones
and the one black square $M_{1, 2002}$ with a one. Fill the rest of the board with zeros. Then there are 3 ones
in rows 1 and 2 with 1 one in each other row. There are 3 ones in column 2002 with 1 one in each other row. In
other words, this arrangements satisfies the problem constraints. However, there are 2005 white unit squares
containing 1, a contradiction to the problem statement.

Alternatively, filling all of the $2001 \times 2003$ top left corner with ones, as well as the bottom-right corner,
satisfies the problem constraints and gives \\
$2\cdot1001^2 \pm 1$ white squares depending on
choice of board orientation; in either case, an odd number. In fact it is difficult to find a board configuration
that satisfies the problem setup but doesn't contain an odd number of white squares, suggesting a typo.

I will prove that both the number of white and black squares must be odd.

Consider how to add (or remove) ones from a good configuration to obtain another good configuration. If you change
a row by 2 ones (this number must be even so that the row sum stays odd), you must change these 2 columns
by 1 one each (so that the column sum stays odd). Furthermore, these last 2 changes must happen in the same row,
so that that row sum stays odd. Extract the 4 squares in question, preserving their orientation, to form a $2 \times 2$
square. These can either be all white (or black), one row (or column) white and the other black, or chessboard-patterned.
In any case, either none of them change (from/to 0 to/from 1) or all of them do, giving either 2 or 4 changes for both
white and black squares. Therefore if one good configuration has an odd number of white and black squares, they all do.

\item
Any binary representation of a positive integer begins with a 1. As a result, excluding
odd-length representations with one more 1 than 0, exactly half of all binary representations have
more ones than zeros.

For the special case of length $2k + 1$ representations with $k+1$ 1s and $k$ 0s, there are
${2k \choose k}$ ways to place the zeros. Because $2003 = \overline{11111010011}_2$, all of these
such 11-digit binary numbers are less than 2003. In addition, every number greater than 2003 has more
1s than 0s in its binary representation. Therefore the total is

$$\frac{2^{11} + \sum_{k=1}^{5} {2k \choose k}}{2} - (2^{11} - 2003 - 1) = 1155$$

\item
Number the vertices 1 to $n$, then consider the edge between vertices 1 and 2. This is part of a single triangle in every
triangulation, and let $k$ be the third vertex of the triangle in a triangulation. The given triangle splits the $n$-gon
into a $(k-1)$-gon and a $(n-k+2)$-gon on either side, where a 2-gon is considered to be a line segment.

With the convention that $a_n$ is the number of triangulations of an $n$-gon, and $a_2 = a_3 = 1$, we can write

$$a_n = \sum_{k=3}^n a_{k-1}a_{n-k+2}$$

This is very reminiscent of the Catalan number recursion, and indeed if we make the substitution $c_n = a_{n+2}$ we get

$$c_n = a_{n+2} = \sum_{k=3}^{n+2} a_{k-1}a_{n+2-k+2} = \sum_{k=0}^{n-1} a_{k+2}a_{n-k+1} = \sum_{k=0}^{n-1} c_k c_{(n-1)-k}$$

This is the Catalan number recursion exactly. As we have the base case covered as well ($a_2 = a_3 = 1$), this sequence is
exactly equal to the Catalan numbers. Therefore $a_n = c_{n-2} = \frac{1}{n-1}{2n-4 \choose n-2}$ is the number of possible
triangulations of a convex $n$-gon.

\item
In Example 8.4, let $k$ be the position of the tallest student when the places are numbered 1 to $n$ from left to right.
Then we cannot have any student in position $i < k$ be taller
than any student in position $j > k$, meaning that this creates a partition where the shortest $(k-1)$ students must be in the
first $(k-1)$ positions, with the rest of the students in positions $(k+1)$ through $n$. Notice, too, that any subarrangement
in these two locations must satisfy the conditions of the problem; therefore there are $a_{k-1}a_{n-k}$ of them.

The tallest student may be in any location, giving $a_n = \sum_{k=1}^n a_{k-1}a_{n-k} = \sum_{k=0}^{n-1} a_ka_{(n-1)-k}$,
as desired.

\item
There are $333 - 33 = 300$ 3-digit numbers divisible by 3, and therefore 900 possible digits to remove.
Using PIE we can count the occurrences of 0, 3, 6, and 9.
Only if one of these digits is removed will a multiple of 3 remain divisible by 3.

There are $3\cdot 34 = 102$ multiples of 3 that begin with a 3, 6, or 9, and $2\cdot4\cdot 30 =  240$ multiples that have
one of these digits in the second place plus those that have it in the third place. The $3 \cdot 4^2 = 48$ numbers where
\emph{every} digit is 0, 3, 6, or 9 are overcounted twice here, but this is appropriate because we want to measure the
odds of removing one of these as a digit.

$$\frac{102 + 240}{900} = \frac{342}{900} = \frac{19}{50}$$

\item
Consider the first 3 draws. Either a pair is found here, with probability
$\frac{{6 \choose 1}\cdot5\cdot2}{{12 \choose 3}} = \frac{3}{11}$, or the game is instantly lost. If a pair is found,
this is the game with 5 pairs of tiles instead of 6. Telescoping, we get our final probability as:

$$\frac{3}{11}\cdot\frac{3}{9}\cdot\frac{3}{7}\cdot\frac{3}{5}\cdot\frac{3}{3} = \frac{9}{385}$$

\item
There are $n!$ bases, and an even number of them are negative. To see why, notice that every element is a part of
$(n-1)!$ bases, and therefore changing a single element will change the sign of $(n-1)!$ bases, an even number if $n \geq 4$.
Let there be $2k$ negative bases, then the sum of all bases is $(n! - 2k) - 2k = n! - 4k \equiv 0 \pmod{4}$ because
$n \geq 4 \implies n! \equiv 0 \pmod{4}$.

\item
Each column consists of two 1s and two -1s, there are ${4 \choose 2}$ ways to do this. Then, partitioning these six
column vectors into 2 groups, with $\vec{v}$ in one column and $-\vec{v}$ in the other, there are 2 cases.

Case 1: Our array is filled with two of the same column vector; the 4 columns are $\{\vec{v}, \vec{v}, -\vec{v}, -\vec{v}\}$.
In this case there are 3 choices of $\vec{v}$ and ${4 \choose 2} = 6$ ways of rearranging them for a total of 18 such arrays.

Case 2: Our array is filled with four different column vectors; the 4 columns in this case must be
$\{\vec{u}, \vec{v}, -\vec{u}, -\vec{v}\}$. There are ${3 \choose 2} = 3$ choices of $\vec{u}$ and $\vec{v}$ and $4! = 24$
ways of rearranging them inside the matrix for a total of 72 such arrays.

Our sum across the two cases is $18 + 72 = 90$.

\item
This incrementing the index of the binomial coefficient by $p$ and asking to prove divisibility by $p$ (i.e. that the sum
is congruent to 0 $\pmod{4}$) is asking for an application of Lucas' Theorem (Theorem 3.3). Writing $n = mp + r$:

\begin{align*}
\sum_{k=0}^{n} (-1)^k{n \choose pk} & \equiv \sum_{k=0}^n (-1)^k{m \choose k}{r \choose 0} \\
& \equiv \sum_{k=0}^m (-1)^k{m \choose k} \\
& \equiv (1 - 1)^m \equiv 0 \pmod{p}
\end{align*}

\item
First, we will note that it isn't possible to form two distinct triangles with 4 segments, so there is no PIE concern.
Second, we will do some basic counting: there are ${10 \choose 2} = 45$ segments, ${45 \choose 4}$ ways of
choosing the 4 distinct segments, and ${10 \choose 3} = 120$ possible triangles to form. The number of ways
a triangle can be chosen with 4 segments is $120 \cdot 42$, as there are 42 segments not in a given triangle.
Therefore our probability is:

$$\frac{120 \cdot 42}{{45 \choose 4}} = \frac{120\cdot42\cdot24}{45\cdot44\cdot43\cdot42} = \frac{16}{473}$$

\item
Let $\{b_k\}$ be the set of integers with $b_k = a_k - k$. Then we have $0 \leq b_1 \leq b_2 \leq \cdots \leq b_m \leq n-m$.
In addition, all $b_k$ are even, so with one more substitution $c_k = \frac{1}{2}b_k$ we get
$0 \leq c_1 \leq \cdots \leq c_m \leq \left\lfloor \frac{n-m}{2} \right\rfloor$. The number of sequences satisfying this
is equal to the number of ways $(m+1)$ nonnegative integers can add to $\left\lfloor \frac{n-m}{2} \right\rfloor$:
${m + \left\lfloor \frac{n-m}{2} \right\rfloor \choose m}$.

\item
\begin{align*}
\sum_{k=0}^n \frac{1}{(n-k)!(n+k)!} & = \frac{1}{(2n)!}\sum_{k=0}^n {2n \choose n+k} \\
& = \frac{1}{2(2n)!}\left({2n \choose n} + \sum_{k=0}^{2n}{2n \choose k} \right) \\
& = \frac{1}{2(n!)^2} + \frac{4^n}{2(2n)!}
\end{align*}

\item
Forty teams play eachother in ${40 \choose 2} = \sum_{k=0}^{39} k$ games, so clearly the teams' scores must
be 0 to 39. If we assign these scores, there is only one way in which the games can go: each team must win against
the team with a lower score and lose to teams with a higher score. As there are 40! ways of assigning the scores,
and each game generates two equally likely possibilities, our probability is:

$$\frac{40!}{2^{{40 \choose 2}}} = \frac{40!}{2^{780}}$$

\item
Let $p_k$ be the probability that the drawing player wins, given that the current congruency class of the sum is $k$, then
we're looking for $p_0$. Considering two draws at a time, we have:

\begin{align*}
p_0 & = \frac{1}{4} + \left(\frac{1}{2}\cdot\frac{1}{4} + \frac{1}{2}\cdot\frac{1}{4}\right)p_1 + \left(\frac{1}{2}\cdot\frac{1}{2} + \frac{1}{4}\cdot\frac{1}{4}\right)p_2 \\
p_1 & = \frac{1}{4} + \left(\frac{1}{2}\cdot\frac{1}{4} + \frac{1}{4}\cdot\frac{1}{4}\right)p_1 + 2\cdot\frac{1}{2}\cdot\frac{1}{4}p_2 \\
p_2 & = \frac{1}{2} + 2\cdot\frac{1}{4}\cdot\frac{1}{4}p_1 + 2\cdot\frac{1}{4}\cdot\frac{1}{4}p_2 \\
\end{align*}

Which, solving this system, gives $p_0 = \frac{101}{166}$ as our desired probability.

\item
The first thing to notice is if the statement is false, the set $\{S(a) : a \in S_n\}$ maps bijectively modulo $n!$ to the
set of all congruency classes $\pmod{n!}$. Then we can count $\sum{a \in S_n}S(a)$ in two ways, first by summing
all congruency classes:

$$\sum_{a \in S_n}S(a) \equiv \sum_{k=0}^{n!} k = \frac{n!(n! - 1)}{2} \equiv \frac{n!}{2} \pmod{n!} \text{, as n! is even.}$$

On the other hand,

\begin{align*}
\sum_{a \in S_n}S(a) & = \sum_{a \in S_n}\sum_{i=1}^n c_i a_i \\
& = \sum_{i=1}^n c_i \sum_{a \in S_n} a_i \\
& = \sum_{i=1}^n c_i (n-1)!\frac{n(n+1)}{2} \\
& = n! \cdot \frac{n+1}{2}\cdot \sum_{i=1}^n c_i \equiv 0 {\pmod n!} \text{, as n is odd.}
\end{align*}

This contradiction means $\{S(a) : a \in S_n\}$ does not contain all congruency classes mod n!, and by the Pigeonhole
Principle we must have at least two permutations $b$ and $c$ such that $S(b) - S(c) \equiv 0 \pmod{n!}$.

\item
Given that there are three $HT$ sequences and four $TH$ sequences, we must start with a $T$ and end with an $H$.
There are otherwise 5 $H$s and 8 $T$s, and if we imagine the sequence before placing the rest of the $H$s:
$TTTTTTTTTH$, we see that in order to have 3 $HT$ and 4 $TH$ occurrences we must distribute the remaining 5
into 4 groups of $H$s (one of them grouped with the last $H$, may be a group of 0). This can be done in the same
number of ways as partitioning 6 into 4 positive integers: ${5 \choose 3} = 10$ ways. The last group must be on the
far right, but the other 3 groups have 8 choices between the 9 $T$s, for a total of

$${5 \choose 3}{8 \choose 3} = 560 \text{ ways in all.}$$

Alternative: Based on the logic above, the underlying structure of our answer is TXHYTXHYTXHYTXHY,
where heads go in the Y spots and tails go in the X spots. This gives the same multiplication as before.

\item
There are ${n+2 \choose 2} = \frac{(n+1)(n+2)}{2}$ ways to split $n$ into 3 nonnegative integers, and this choice is
independent across the 3 rows. Therefore the total number is ${n+2 \choose 2}^3$ matrices.

\item
Because $P(\text{win}) = P(\text{lose}) = 1/3$, the wins and losses are symmetric, and it will be easier to count this
as $\frac{1}{2}(1 - P(\text{wins = losses}))$. Each outcome is equally likely, and so:

\begin{align*}
P(\text{wins = losses}) & = \sum_{k=0}^3 {6 \choose 2k; 3-k}\left(\frac{1}{3}\right)^6 \\
& = \frac{1}{3^6}\sum_{k=0}^3 {6 \choose 2k}{6 - 2k \choose 3-k} \\
& = \frac{1}{3^6}(20 + 15\cdot6 + 15\cdot2 + 1) \\
& = \frac{47}{243}
\end{align*}

And therefore our desired probability is

$$P(\text{wins}>\text{losses}) = \frac{1}{2}\left(1 - \frac{47}{243}\right) = \frac{98}{243}$$

\item
This is a very rough argument, and worth coming back to to prove rigorously. The maximum density of diagonal elements
in a semiperimeter $n$ submatrix of an $n \times n$ matrix is $1/\left\lfloor\frac{n+1}{2}\right\rfloor$, which comes
from making that submatrix as ``square'' as possible. Noticing the inverse relationship between semiperimeter
and diagonal element density, and the fact that the submatrix of overlap between two semiperimeter $n$ submatrices
will always have a smaller semiperimeter, by PIE the subset $A \cup B$ has lower density than either of the submatrices
$A$ or $B$. Therefore the upper limit on diagonal element density for a union of submatrices (such as the collection
we're interested in) is $1/\left\lfloor\frac{n+1}{2}\right\rfloor$, and therefore our collection must have at least
$n\left\lfloor\frac{n+1}{2}\right\rfloor \geq \frac{n^2}{2}$ squares in it.

\item
After satisfying the minimum requirements, there are $k$ places to put $n-(m+1)k$ empty chairs. This is equivalent
to partitioning $n-(m+1)k$ into $k$ nonnegative integers, which we can do in ${n-mk-1 \choose k-1}$ ways.

Then there are $n$ ways of rotating this arrangement, and we overcount the students' placement $k$ times, so our
final answer is $\frac{n}{k}{n-mk-1 \choose k-1}$.

\item
I'm going to include degenerate triangles in my count, that is, triangles with $a = 0$ or $a+b = c$. A triangle with
integer side lengths including $k \leq m$ and $2m + 1 - k$ can have a third side between $2(m-k) + 1$ and $2m + 1$,
i.e. there are $2k + 1$ of them. The only overlap between the triangles of different $k$ values is when the triangle
is isosceles (note there are no such equilateral triangles).
If $k \geq 2(m-k)+1 \iff \left\lceil\frac{2m + 1}{3}\right\rceil \leq k \leq m$, there are
$m - \left\lceil\frac{2m + 1}{3}\right\rceil + 1$ such isosceles triangles. On the other hand, all $m$ possible long-sided
isosceles triangles are valid.
Therefore

$$n = \left(\sum_{k=0}^m 2k+1\right) - (m - \left\lceil\frac{2m + 1}{3}\right\rceil + 1) - m = m^2 - 1 + \left\lceil\frac{2m + 1}{3}\right\rceil$$

\item
Bonnie's probability of winning the first game is half of Alfred's, so \\
$P(\text{win game }|\text{ go first}) = 2/3$. If we call $p_n$ the probability that Alfred wins game $n$, we have the
recursion

$$p_n = \frac{1}{3}p_{n-1} + \frac{2}{3}(1 - p_{n-1}) = \frac{2 - p_{n-1}}{3}; \; p_1 = \frac{2}{3}$$

Calculating it out to find $p_6$, we get:

\begin{align*}
& p_2 = \frac{\frac{6 - 2}{3}}{3} = \frac{4}{9} \\
& p_3 = \frac{\frac{18 - 4}{9}}{3} = \frac{14}{27} \\
& p_4 = \frac{\frac{54 - 14}{27}}{3} = \frac{40}{81} \\
& p_5 = \frac{\frac{162 - 40}{81}}{3} = \frac{122}{243} \\
& p_6 = \frac{\frac{486 - 122}{243}}{3} = \frac{364}{729} \text{ is our final answer.}
\end{align*}

\item
For a particular element $a \in \{1, 2, \ldots, 1998\}$ it is absent in half of all $A_i$. Therefore it is absent in $1/2^n$
of all $\cup_{i=1}^n A_i$, or $2^{1997n}$ of them to be precise. Then we can find the sum of all such unions by
summing the size of the set and subtracting the missing elements:

$$\sum_{(A_1, \ldots, A_n) \in F} |\cup_{i=1}^n A_i| = 1998\cdot 2^{1998n} - 1998 \cdot 2^{1997n} = 1998\cdot 2^{1997n}(2^n - 1)$$

\item
\begin{align*}
\sum_{k=0}^n \frac{1}{k+1}{n \choose k} & = \sum_{k=0}^n \frac{1}{n+1}{n+1 \choose k} \\
& = \frac{1}{n+1}\left( \sum_{k=0}^{n+1} {n+1 \choose k} - 1 \right) \\
& = \frac{1}{n+1}\left(2^{n+1} - 1\right)
\end{align*}

\item
Assume, to achieve a contradiction, that there is an element $a_i$ that appears in $P_r, P_s, P_t$. Then there are two
subsets, say $P_r$ and $P_s$, with $r \neq i, s \neq i$. Since $P_r \cap P_s \neq \emptyset$, we must have
$P_r = \{a_r, a_s\}$ or $P_s = \{a_r, a_s\}$. This is a contradiction, since $a_i \in P_r, a_i \in P_s$, and therefore there
is no element which appears in more than two of the $P_k$s.

As there are $n$ subsets with two elements each, there are $2n$ elements in all of the $P_k$s. Since no element
may appear 3 or more times, there are $2n$ elements in all the subsets if and only if each element appears twice.

\item
There is a one-to-one bijection between legal Chomp board states and paths from (0, 5) to (7, 0) which only move
downward and rightward. To construct the path from the Chomp board state, take the continuation of the left
side of the board up to (0, 5) and the bottom of the board to (7, 0) and consider the right-and-top edge path between
these two points. To construct the Chomp board state from the path, everywhere the path moves down take a bite
of the square immediately to the right. Therefore there are ${12 \choose 5} = 792$ paths.

\item
We can partition the 15 points into 3 sets of 5 points which are only combos with each other: $\{a_1, a_4, \ldots, a_{13}\}$,
$\{a_2, a_5, \ldots, a_{14}\}$, and $\{a_3, a_6, \ldots, a_{15}\}$. Then any combo is simply a pair of adjacent elements in one
of these 3 subsets. There is 1 way of choosing 0 elements from one of these subsets, 5 ways of choosing 1 element,
and 5 ways of choosing 2 elements without a combo (i.e. without a pair of adjacent elements, exactly it is
$\frac{5}{2}{2 \choose 1}$ c.f. Problem 9.71) for a total of 11 ways of choosing elements from a subset. These choices
are independent between the 3 subsets, so there are $11^3 = 1331$ total ways to pick sets of points without a combo.

\item
For an integer $k$ in this range, there are $k-1$ integers less than it and $10-k$ integers greater than it; therefore
it will contribute $k(k-1-(10-k)) = k(2k-11)$ among all its pairs. Then our answer is

$$\frac{1}{9}\sum_{k=1}^{10}k(2k-11) = \frac{165}{9} = \frac{55}{3}$$

N.B. There are 9 pairs of integers in each permutation sum, 10! permutations, and ${10 \choose 2} = \frac{10 \cdot 9}{2}$
pairs of integers in all. Therefore each pair of integers appears $9!/2$ times in the sum.

\item
\begin{align*}
\sum_{k=m}^n (-1)^{k+m}{n \choose k}{k \choose m} & = \sum_{k=m}^n (-1)^{k+m}{n \choose n-k; m; k-m} \\
& = \sum_{k=m}^n (-1)^{k+m}{n \choose m}{n-m \choose k-m} \\
& = {n \choose m}\sum_{k=0}^{n-m}(-1)^k{n-m \choose k} \\
& = {n \choose m}(1-1)^{n-m} \\
& = {n \choose m}0^{n-m} = \delta_{n,m} \text{, the Kronecker delta.}
\end{align*}

\item
First of all, $n \geq b_{k+1} \geq b_k + a_k \geq \ldots \geq b_1 + \sum_{i=1}^k a_i \geq 1 + \sum_{i=1}^k a_i$
is the requirement for such a sequence to exist at all. Secondly the function

$$f(b_1, \ldots, b_{k+1}) = (0, b_2 - b_1 - a_1, \ldots, b_j - b_1 - \sum_{i=1}^{j-1}a_i, \ldots, b_{k+1} - b_1 - \sum_{i=1}^k a_i$$

forms a bijection between the set of our solution sequences and the set of all nondecreasing sequences of length $k+1$
starting at 0 and ending at $m, 1 + \sum_{i=1}^k a_i \leq m \leq n$. The number of these sequences is equal to the number
of ways to partition $m$ into $k$ nonnegative integers: ${m + k - 1 \choose k-1}$. Therefore, letting $A = \sum_{i=1}^k a_i$,
our answer is

$$\sum_{m=A + 1}^n {m + k - 1 \choose k-1} = {m + n \choose k} - {m + A + 1 \choose k}$$

\item
This is supposed to read $x \in A \cup B$ instead of $x \in A \cap B$, because if $A$ and $B$ are disjoint this
intersection is empty.

Lemma 1: If $(x_1, x_2, \ldots, x_{|A| + |B| + 1})$ is a sequence in $A \cup B$ satisfying
$x_{k+1} \in \{x_k + a \in A, x_k - b \in B\}, x_{k + 1} \neq x_{k-1}$ unless by necessity (i.e. $a=b$ and these are the
only choices), then there is some $k \neq 1 : x_k = x_1$.

Proof of Lemma 1: By the problem statement such a sequence always exists. There are only $|A| + |B|$ elements in $A \cup B$,
so by the Pigeonhole principle we must have a repeat element. Let's say, to achieve a contradiction, that this is $x_m$
and that it appears before $x_1$ appears again (if it ever \emph{does} repeat). There are only two elements, $x_m - a$ and
$x_m + b$, that can precede $x_m$ in the sequence. They must appear on either side of $x_m$ the first time it appears,
and therefore $x_{m-1}$ repeats before $x_m$, a contradiction.

Call the sequence $(x_1, \ldots, x_k)$ up to the first repeat of $x_1$ a \emph{cycle}.

Lemma 2: These \emph{cycles} form a partition of $A \cup B$.

Proof of Lemma 2: By the problem statement, every element belongs to a cycle. In addition, we cannot have an
element belong to 2 different cyles. If we did, then we could start in one cycle, complete half of it, switch over to the
other cycle, and have a repeat element earlier than $x_1$.

In a cycle, let $n_a, n_b$ be the number of elements in $A, B$ respectively. Then we add $a$ $n_a$ times and subtract
$b$ $n_b$ times to end where we started; that is, $a\cdot n_a - b\cdot n_b = 0 \iff a\cdot n_a = b \cdot n_b$. Summing
across all cycles, $a|A| = b|B|$.

\item
See the answer to Problem 9.16 for a more in-depth treatment of this sum.

\item
We know that letters 1-7 have been distributed already, and therefore must be in increasing order. There are $2^7$
ways of this happening, and if there are $k$ letters left to type after lunch, the $9^{\text{th}}$ letter may be placed
at any time among them in $k+1$ ways, or it may have already been typed (that is, it may not appear after lunch).
Therefore, the total number of after lunch typing orders is

$$\sum_{k=0}^7 {7 \choose k}(k+2) = 7\cdot2^6 + 2\cdot2^7 = 704$$

\item
We'll use a linear recursion to solve for the number of $n$-digit numbers satisfying this property and then plug in $2n$
instead. Let $a_n$ be the number of $n$-digit even-sum numbers starting with an even number, $b_n$ the even-sum
starting with odd, $c_n$ the odd-sum starting with even, and $d_n$ the odd-sum starting with odd. Then we can only
get $a_n$ from $a_{n-1}$ and $b_{n-1}$ (and similar conditions exist for $b_n, c_n, d_n$), and there are 4 even
numbers and 5 odd numbers we can start with. This gives:

$$\begin{bmatrix} a_{n+1} \\ b_{n+1} \\ c_{n+1} \\ d_{n+1} \end{bmatrix} = \begin{bmatrix} 4 & 4 & 0 & 0 \\ 5 & 0 & 0 & 5 \\ 0 & 0 & 4 & 4 \\ 0 & 5 & 5 & 0 \end{bmatrix} \begin{bmatrix} a_n \\ b_n \\ c_n \\ d_n \end{bmatrix}; \; \begin{bmatrix} a_1 \\ b_1 \\ c_1 \\ d_1 \end{bmatrix} = \begin{bmatrix} 4 \\ 5 \\ 4 \\ 5 \end{bmatrix}$$

After the eigendecomposition and solving the initial value problem, we get formulas for $a_n$ and $b_n$ which we can
combine into:

\begin{align*}
a_{2n} + b_{2n} = & \; \frac{9^{2n}}{2} + \left(\frac{-19 + \sqrt{161}}{4\sqrt{161}}\right)\left(\frac{1+\sqrt{161}}{2}\right)^{2n} \\
& + \left(\frac{19 + \sqrt{161}}{4\sqrt{161}}\right)\left(\frac{1-\sqrt{161}}{2}\right)^{2n}
\end{align*}

This is not satisfying as an answer, but I have yet to come up with another way of solving this problem or identifying
the sequence this produces: $\{56, 3736, 277816, 21774296, 1744365176, \ldots \}$. In any case, 9 is the largest eigenvalue
and therefore $\lim_{n \to \infty} \frac{a_{2n} + b_{2n}}{\left(\frac{9^{2n}}{2}\right)} = 1$.

\item
There are $n$ ways to choose the first marble, then there are $n-1-a_1$ ways to pick the second (without picking
the original marble or the $a_1$ marbles in front of it), $n-2-a_2$ ways to pick the third, and $n-i-a_i$ ways to pick
the $i^{\text{th}}$.

For any particular choice of $\{a_i\}$, this gives $n \prod_{i=1}^n n - i - a_i$ ways in total, or 0 ways if $a_i + i \geq n$ for
any $i$.

\item
If $n$ is even, $\left \lfloor \frac{n}{2} \right \rfloor = \frac{n}{2}$ and this is half of the sum across the $n^{\text{th}}$
row of binomial coefficients. If $n$ is odd, $\left \lfloor \frac{n}{2} \right \rfloor = \frac{n-1}{2}$ and this is half of the
sum across the row, but missing the element $\left[{n \choose \frac{n+1}{2}} - {n \choose \frac{n-1}{2}}\right]^2 = 0$.
Therefore in either case, we have

\begin{align*}
\sum_{k=0}^{\left\lfloor \frac{n}{2} \right\rfloor} \left[{n \choose k} - {n \choose k-1}\right]^2 & = \frac{1}{2}\sum_{k=0}^{n+1} \left[{n \choose k} - {n \choose k-1}\right]^2 \\
& = \frac{1}{2}\sum_{k=0}^{n+1} \left[{n \choose k}^2 + {n \choose k-1}^2 - 2{n \choose k}{n \choose k-1}\right] \\
& = \frac{1}{2}\sum_{k=0}^n {n \choose k}^2 + \frac{1}{2}\sum_{k=1}^{n+1}{n \choose k-1}^2 - \sum_{k=0}^{n+1}{n \choose k}{n \choose k-1} \\
& = \sum_{k=0}^n {n \choose k}^2 - \sum_{k=0}^{n+1}{n \choose k}{n \choose n-k+1} \\
& = {2n \choose n} - {2n \choose n+1} \text{ (Vandermonde's Identity)} \\
& = {2n \choose n} - \frac{n}{n+1}{2n \choose n} \\
& = \frac{1}{n+1}{2n \choose n}
\end{align*}

\item
There are $15!$ ways to choose the boys, the same number of ways to choose the girls, and $2^{15}$ ways of
arranging the pairs. When we account for the $15$ ways this table can be permuted, there is also the problem
of double counting the alternating arrangements BGBG...BG and GBGB...GB, so we must
subtract one from the ways of arranging the pairs to account for this. Then our final answer is $15!14!(2^{15} - 1)$
possible seating arrangements.

\item
By noticing that the requirements on $\{a_i\}$ are similar to Example 5.11, we can create a bijection between
series satisfying the given constraints and those of Example 5.11. Let $c_k = a_{n+1 - k}$, then the first
requirement becomes $1 \leq a_{n + 1 - k} = c_k \leq k$, which forces $c_1 = 1$. The second requirement is
$a_{i+1} \geq a_i + 1 \iff c_i \leq c_{i+1} + 1$, and therefore the constraints are identical to Ex. 5.11. Because
multiplication is commutative, $\sum (a_1 \cdots a_n) = \sum (c_1 \cdots c_n) = (2n-1)!!$.

\item
This is Moser's circle problem, and we will get to the answer by using Euler's Formula: $V - E + F = 1$ for a
compact subset of $\mathbb{R}^2$. There are 2 types of vertices, those on the boundary of the circle ($V_{\text{b}}$)
and those in the interior ($V_{\text{i}}$). Similarly, there are 2 types of edges: $E_{\text{b}}$ for circular arcs
and $E_{\text{i}}$ for line segments.
As $V_{\text{b}} = E_{\text{b}} = n$, we have $F_n = 1 + E_{\text{i}} - V_{\text{i}}$.

Assuming no coincidence of interior diagonals, every 4 boundary vertices correspond bijectively to each interior
point. Therefore $V_{\text{i}} = {n \choose 4}$.

There are ${n \choose 2}$ interior diagonals, and each one is further split into connections between interior points.
There are ${n \choose 4}$ interior points, each of which has 4 connections with its neighbors. Those connections
are shared between 2 vertices, so the total here is $E_{\text{i}} = {n \choose 2} + 2{n \choose 4}$. Therefore

$$F = 1 + {n \choose 2} + 2{n \choose 4} - {n \choose 4} = 1 + {n \choose 2} + {n \choose 4}$$.

This can be written as ${n \choose 0} + {n \choose 2} + {n \choose 4}$, and the pattern becomes even cleaner
if we rewrite it as the sum of previous elements in Pascal's Triangle: ${n-1 \choose 0} + {n-1 \choose 1} + {n-1 \choose 2}
+ {n-1 \choose 3} + {n-1 \choose 4} = \sum_{i=0}^4 {n-1 \choose i}$, giving Moser's circle sequence and simultaneously
explaining why $F_n = 2^{n-1}$ for \\ $n \in \{1, 2, 3, 4, 5\}$ but not $n = 6$.

\item
For each element of $S$, there are ${4 \choose 2} = 6$ ways we can choose exactly two subsets for it to belong to.
However, if we have a set appear more than once in our 4-double cover, this will overcount, e.g. treating $\{A, A, B, C\}$
and $\{A, B, C, A\}$ as different double covers.

If two of our subsets are equal in the 4-double cover, the other two subsets must be exactly their complement.
There are ${4 \choose 2} = 6$ ways of rearranging $\{A, A, A^c, A^c\}$, so each is overcounted 5 times. There are
$|\mathcal{P}(S)| = 2^n$ ways of picking $A$, and each subset is double counted because it appears once as $A$ and once as $A^c$,
so there are $2^{n-1}$ of these such repeat 4-double covers. In total, there are

$$6^n - 5\cdot 2^{n-1} \text{ four-double covers in all.}$$

\item

\item
The degree of any $n$ (i.e. number of dominoes including $n$) must be even, except possibly excluding endpoints,
but there are 39 unordered dominoes that include $n$ in our set. This means the maximum is $\frac{38\cdot 40 + 2}{2} = 761$,
and it just remains to show that it is possible. This can be done inductively, starting with all dominoes including 1 except
for (1, 40) and ending with (1, 2), then doing the same with 2 excluding (2, 40) and ending with (2, 3), etc. This will give
$2\sum_{i=0}^{19} (38 - 2i) = 760$, and then we can place (40, 1) at the very beginning or the very end to get 761 total.

\item
\begin{align*}
S(m, n) & = \sum_{k=0}^{n-1} (-1)^k {n \choose k}(2^{n-k} - 1)^m \\
& = \sum_{k=0}^{n-1} (-1)^k {n \choose k}\sum_{i=0}^m (-1)^{m-i} {m \choose i} 2^{i(n-k)} \\
& = \sum_{k=0}^{n-1} \sum_{i=0}^m (-1)^{k+m-i} {n \choose k} {m \choose i} 2^{i(n-k)} \\
& = \sum_{i=0}^{m} (-1)^{m-i} {m \choose i} 2^{in} \sum_{k=0}^{n-1} (-1)^k {n \choose k} 2^{-ik} \\
& = \sum_{i=0}^{m} (-1)^{m-i} {m \choose i} 2^{in} \left[ \left( 1 - 2^{-i} \right)^n + (-1)^{n+1}2^{-in} \right] \\
& = \sum_{i=0}^{m} (-1)^{m-i} {m \choose i} \left[ \left( 2^i - 1 \right)^n + (-1)^{n+1} \right] \\
& = \sum_{i=0}^{m} (-1)^i {m \choose i} \left[ \left( 2^{m-i} - 1 \right)^n + (-1)^{n+1} \right] \\
& = \sum_{i=0}^{m} (-1)^i {m \choose i} \left( 2^{m-i} - 1 \right)^n + (-1)^{n+1}\sum_{i=0}^{m} (-1)^i {m \choose i} \\
& = \sum_{i=0}^{m} (-1)^i {m \choose i} \left( 2^{m-i} - 1 \right)^n + (-1)^{n+1}(1 - 1)^m \\
& = \sum_{i=0}^{m - 1} (-1)^i {m \choose i} \left( 2^{m-i} - 1 \right)^n - (-1)^m \left(2^0 - 1)^n\right) \\
& = \sum_{i=0}^{m - 1} (-1)^i {m \choose i} \left( 2^{m-i} - 1 \right)^n \\
& = S(n, m)
\end{align*}

\item
Let $f_n$ be the process of removing every other card in a stack of $n$ cards starting with the top card, and let $g_n$
be a similar process but skipping the first card. Then there are $\left\lfloor \frac{n}{2} \right\rfloor$ elements
remaining after $f_n$ and $\left\lfloor \frac{n+1}{2} \right\rfloor$ elements remaining after $g_n$. In addition,
the top card will be taken if the previous process was $f_n : n$ even or $g_n : n$ odd. Therefore our sequence is

$$f_{2000}\cdot f_{1000}\cdot f_{500}\cdot f_{250}\cdot f_{125}\cdot g_{62}\cdot g_{31}\cdot f_{16}\cdot f_{8}\cdot f_{4}\cdot f_{2}\cdot f_{1}$$

Working backwards, position $k$ in $f_n$ was previously in position $2k - 1$ and passed elements were in position $2k$.
This is reversed for $g_n$.
Therefore, 1999 being in position 1 of $f_2$ means its sequence goes:

$$1 \to 2 \to 4 \to 8 \to 15 \to 29 \to 58 \to 116 \to 232 \to 464 \to 928$$

Therefore there were 927 cards above the card labeled 1999.

\item
This is a multinomial counting problem; the number of partitions is given by

$$|P| = {|S| \choose n_1; 2n_2; \ldots; (k-1)n_{k-1}} = \frac{n!}{\prod_{i=1}^k (in_{i})!}$$

\item
The sum on the left is the total number of fixed points in the symmetry group $S_n$, so we either need to establish
a bijection or prove that both sums are equal to the same known quantity.

Using Fubini's Principle, we can count the number of fixed points in another way. Let 
$S = \{S(i, \sigma) : i \text{ is a fixed point of } \sigma$. Then $\sum_{k=0}^n kp_n(k) = \sum_{\sigma \in S_n} |S(*, \sigma)|
= \sum_{i =1}^n |S(i, *)|$.
For a particular $i$, there are $n$ places for it to be mapped to, all with equal likelihood. Therefore it will be a
fixed point in $|S(i, *)| = \frac{1}{n}|S_n| = \frac{n!}{n} = (n-1)!$ of the permutations in $S_n$, and
$\sum_{i =1}^n |S(i, *)| = \sum_{i=1}^n (n-1)! = n(n-1)! = n!$.

Continuing with this method, we only need to prove $\sum_{k=0}^n (k-1)^2 p_n (k) = n!$. First, note that
$\sum_{k=0}^n {k \choose i} p_n (k)$ is the number of ways to pick $i$ elements from the $k$ fixed points
The expression
$\sum k {n \choose k} p_{n-k} (0)$ is the number of ways of  

\item
Let's break this into 3 cases, where there are 0, 1, or 2 mail-getting houses on the ends of the street. In all of these
cases, let $k$ be the number of non-mail-getting houses and $i$ be the number of adjacent pairs of these.

Case 1: Both ends of the street don't get mail. There
are $k-1$ houses getting mail, and $2k-1 \leq 19 \leq 3k-1 \iff 7 \leq k \leq 10$. In addition, $2k - 1 + i = 19 \iff
i = 20 - 2k$.

Case 2: One end of the street gets mail. There are $k$ houses
getting mail. Then $2k \leq 19 \leq 3k \iff 7 \leq k \leq 9$, and $i = 19 - 2k$.

Case 3: Both ends of the street get mail. There are $k+1$ houses getting mail.
Then $2k+1 \leq 19 \leq 3k+1 \iff 6 \leq k \leq 9$, and $i = 18 - 2k$.

In all of these cases, there are $i$ instances of pairs of houses not getting mail, which can be chosen in ${k \choose i}$
many ways. This totals up across the 3 cases to

\begin{align*}
& \sum_{k=6}^{10} {k \choose 20-2k} + 2{k \choose 19-2k} + {k \choose 18-2k} \\
= & \sum_{k=6}^{10} {k+2 \choose 20-2k} = 351
\end{align*}

This is the Padovan sequence; Cases 3, 2, and 1 sum to $P_{21}$, $2P_{22}$, and $P_{23}$, respectively; all together
they sum to $P_{27}$. Under the recursion $P_n = P_{n-2} + P_{n-3}$ the case sums are equivalent to
$P_{24} + P_{25} = P_{27}$, as expected. In general, with $n$ houses on the street we will get $P_{n+8}$ different
possible patterns of mail delivery.

\item
An \emph{interesting} coloring is counted 3 times, one for each direction it can face. For a coloring that is interesting
along 2 axes this overcounts by 1, and for an interesting coloring along 3 axes this overcounts by 2.

\item
For $a + b = c + d = k \leq 2m + 1$, there are $\left \lfloor \frac{k-1}{2} \right \rfloor$ pairs that sum to $k$. For $k > 2m+1$, we
can perform the transformation $f(a) = 2m + 1 - a$ to get $a + b = k \iff f(a) + f(b) = 4m + 2 - k < 2m+1$, meaning that there
are $\left \lfloor \frac{4m + 1 - k}{2} \right \rfloor$ pairs that sum to $k$. A particular case is $k = 2m+1$, because this is
the only instance of $\left \lfloor \frac{k-1}{2} \right \rfloor = m$. Assuming there can be no repeat elements among the set $\{ a, b, c, d \}$, there can be no overlap of the sets by
$a + c = b + d$ or $a + d = b + c$:

\begin{align*}
& a + b = c + d \\
& a + c = b + d \implies b - c = c - b \iff b = c \text{, or} \\
& a + d = b + c \implies b - d = d - b \iff b = d
\end{align*}

Our total number of sets is then

\begin{align*}
4 \cdot \sum_{n = 2}^{m-1} {n \choose 2} + {m \choose 2} & = 4{m \choose 3} + {m \choose 2} \\
& = {m + 1 \choose 3} + 3{m \choose 3} \text{, if you would prefer this form.} \\
\end{align*}

\item

\item


Let $s_n$ be the number of ways to tile a $2 \times n$ chessboard with the given boards, and for a specific tiling,
let $k < n$ be the value of the top square just to the left of the rightmost place where the boards on the top
and bottom terminate together.

We must now determine the number of tilings of a $2 \times n$ board that satisfy $k=0$.
Each end must have a $1 \times 2$ in one row and a $1 \times 3$ in the other,
with one row being one square ahead of the other. Call the number of tilings of these $1 \times n-2$ and
$1 \times n-3$ rows $m_n$. Then we may put a $1 \times 2$ board in the shorter row in $m_{n-1}$ many ways,
or a $1 \times 3$ board in each row in $m_{n-3}$ many ways, giving the recursion (only valid for $n \geq 7$):

$$m_n = m_{n-1} + m_{n-3}; \; (m_i) = (0, 0, 1, 1, 0, 2, 2) \text{ for } i = 0, 1, 2, 3, 4, 5, 6$$

This recursion is solved with the generating function

\begin{align*}
m_n = [x^n]f(x) & = [x^n]\left(x^2 + x^3 + \frac{2x^5}{1 - x - x^3}\right) \\
& = [x^n]\left(2 - x^2 + x^3 - \frac{2(1 - x - x^2)}{1 - x - x^3}\right)
\end{align*}

The tiling to the left of $k$ is a representative of $s_k$, and the tiling to the right of $k$ is a representative of
$m_{n-k}$. Therefore, calling $g(x)$ the generating function for $\{ s_n \}$, we can write

\begin{align*}
s_n & = [x^n]g(x) \\
& = \sum_{k=0}^{n-1} s_k m_{n-k} \\
& = [x^{n-1}] (g(x) \cdot xf(x))
\end{align*}

\item

\item

\item

\item

\item

\item

\item

\item

\item

\end{enumerate}

\bibliographystyle{plain}
\bibliography{references}

\end{document}